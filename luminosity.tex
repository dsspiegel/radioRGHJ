%\documentclass[12pt,preprint]{aastex}
\documentclass[12pt]{emulateapj} 
\usepackage{amsmath}
\usepackage{amssymb}
\usepackage{amstext}
%\usepackage{apjfonts}
\usepackage{graphicx}
\usepackage{epstopdf}
\usepackage{epsfig}
\usepackage{url}
%------------------------------------------
% User-defined macros.
\usepackage{color}


\definecolor{myColor}{rgb}{0.9,0.9,0.9}  
%------------------------------------------
%%%%%%%%%%%%%%%%%%%%%%%%%%%%%%%%%%%%%%%%%%%%%%
\begin{document}
\renewcommand\bottomfraction{.9}
\newcommand{\PD}[2]{\frac{\partial {#1}}{\partial {#2}}}

\title{Luminosity function of RGHJs}
\author{Yuka Fujii\altaffilmark{1}}
\altaffiltext{1}{Institute of Astronomy, Madingley Road, University of Cambridge, Cambridge CB3 0HA, United Kingdom}

\begin{abstract}
N/A
\end{abstract} 

\keywords{planetary systems --- planets and satellites: general}

%%%%%%%%%%%%%%%%%%%%%%%%%%%%%%%%%%%%%%%%%%%%%%%%%%%%%%%
\section{Luminosity Function Note \#1} 
\label{sec:luminosity function}

Assume:
%%%
\begin{displaymath}
\frac{dn}{dL} = \left\{
\begin{array}{ll}
C' L^{-\alpha } & (L_{\rm min} < L < L_{\rm max}) \\
0 & (\mbox{elsewhere})
\end{array}
\right.
\end{displaymath}
%%%
where $C' $ is the proportional coefficient. 

%%%
\begin{equation}
\frac{d^2n}{dL dr} dL dr =  4 \pi C L^{-\alpha } r^2 dL dr \equiv \Psi (L, r) dL dr
\end{equation}
%%%

The relationship between flux $F$ and Luminosity $L$ is:
%%%
\begin{equation}
F = \frac{L}{r^2}
\end{equation}
%%%
where $r$ is the distance to the target. 

The number of the targets within distance $R$ which have flux at $F \sim F+dF$ is
%%%
\begin{eqnarray}
\frac{dn}{dF} dF &=& \int dr \int dL \; \Psi (L, r) \\
&=& \int _0^{R} dr \int _{F}^{F+dF} dF \; \Psi (F r^2, r)\; det(J) 
\end{eqnarray}
%%%
where
%%%
\begin{equation}
det(J) = \left| \PD{r}{r}\PD{L}{F} - \PD{r}{F}\PD{L}{r} \right| = r^2
\end{equation}
%%%
Therefore, 
%%%
\begin{eqnarray}
\frac{dn}{dF} dF &=& \left[ 4 \pi \int _0 ^{R} dr \; r^4 \Phi (F, r) \right] dF\\
\Phi (F, r) &=& \left\{
\begin{array}{ll}
C (F r^2)^{-\alpha } & (\sqrt{L_{\rm min}/F} < r < \sqrt{L_{\rm max}/F}) \\
0 & (\mbox{elsewhere})
\end{array}
\right. 
\end{eqnarray}
%%%
Therefore, 
%%%
\begin{eqnarray}
&& \frac{dn}{dF} dF \\
&=& 0 \;\;\;\; (R < \sqrt{L_{\rm min}/F}) \\
&=& \frac{4\pi C F^{-\alpha }}{5 - 2\alpha} \left[ R^{5 - 2\alpha} - \left( \sqrt{ \frac{L_{\rm min}}{F}} \right)^{5 - 2\alpha} \right] \\
&& \;\;\;\;\;\;\;\;\;\;\;\;\;\;  (\sqrt{L_{\rm min}/F} < R < \sqrt{L_{\rm max}/F} ) \\
&=& \frac{4\pi C F^{-\alpha }}{5 - 2\alpha} \left[ \left( \sqrt{ \frac{L_{\rm max}}{F} } \right)^{5 - 2\alpha} - \left( \sqrt{\frac{L_{\rm min}}{F}} \right)^{5 - 2\alpha } \right] \\
&& \;\;\;\;\;\;\;\;\;\;\;\;\;\; (\sqrt{L_{\rm max}/F} < R) 
\end{eqnarray}
%%%



\vspace{2\baselineskip}



\end{document}



The probability to have the maximum luminosity of $F \sim F + dF$, $\mathcal{P} (F_{\rm max})$, with $N$ samples is
% $expected value of the maximum luminosity $L_{\rm max}$ is
%%%
\begin{eqnarray}
\mathcal{P}(F_{\rm max}) &\cdot & F_{\rm max} \\
&=& N \cdot p (F_{\rm max}) \cdot P^{N-1}(<F_{\rm max}) \cdot dF_{\rm max}\end{eqnarray}
%%%

%%%%%%%%%%%%%%%%%%%%%%%%%%%%%%%%%%%%%%%%%%%%%%%%%%%%%%%
\section{Luminosity Function Note \#2} 
\label{sec:luminosity function}

Assume 
%%%
\begin{equation}
\frac{dn}{dL} = L^{-\alpha }
\end{equation}
%%%
Probability for a star to have luminosity $L\sim L+dL$ is 
%%%
\begin{equation}
p (L) dL = C L^{-\alpha }
\end{equation}
%%%


Probability for a star to have luminosity larger than $L$ is 
%%%
\begin{equation}
P(>L) = \int _L^{\infty} C L^{-\alpha } dL = C \frac{L^{-\alpha +1}}{-\alpha + 1}
\end{equation}
%%%

Probability for a star to have luminosity less than $L$ is 
%%%
\begin{equation}
P(<L) = 1 -P (>L) \end{equation}
%%% 

The probability to have the maximum luminosity of $L \sim L + dL$, $\mathcal{P(L_{\rm max}})$, with $N$ samples is
% $expected value of the maximum luminosity $L_{\rm max}$ is
%%%
\begin{eqnarray}
\mathcal{P}(L_{\rm max}) &\cdot & dL_{\rm max} \\
&=& N \cdot p (L_{\rm max}) \cdot P^{N-1}(<L_{\rm max}) \cdot dL_{\rm max} \\
&=& N C L_{\rm max}^{-\alpha } \left[ 1 - \left( \frac{CL_{\rm max}^{-\alpha+1}}{-\alpha+1} \right) \right]^{N-1}
\end{eqnarray}
%%%
Thus, the expected value for the maximum luminosity is
%%%
\begin{eqnarray}
< L_{\rm max} > (N) &=& \int L_{\rm max} \cdot \mathcal{P_{\rm max}}(L_{\rm max}) \cdot dL_{\rm max} \\
&=& N C \int dL_{\rm max} \cdot L_{\rm max}^{-\alpha +1} \left[ 1 - \left( \frac{CL_{\rm max}^{-\alpha+1}}{-\alpha+1} \right) \right]^{N-1} \\
&\equiv & N C \int dL_{\rm max} \cdot L_{\rm max}^{-\alpha +1} K^{N-1}(L_{\rm max})
\end{eqnarray}
%%%

%Thus, the expected value for the maximum flux from the spherical shell at $l \sim l+dl$, $f_{\rm max} (l)$, is:
%%%
%\begin{equation}
%f_{\rm max} (l) = < L_{\rm max} > (4\pi l^2 dl) / l^2
%\end{equation}
%%%


As we observer further (i.e. while the distance $l$ increases by d$l$), the maximum luminosity increases
%%%
\begin{eqnarray}
&& \frac{d < L_{\rm max} >}{dl} = \frac{dN}{dl} \frac{d < L_{\rm max} >}{dN} \\
&\propto & l^{2} \left[ \int dL_{\rm max} \cdot L_{\rm max}^{-\alpha +1} K^{N-1}(L_{\rm max}) \right. \\
&&\left. + N (N-1) \int dL_{\rm max} \cdot L_{\rm max}^{-\alpha +1} K^{N-2}(L_{\rm max}) \right]
\end{eqnarray}
%%%



Let us compute the expected value for the maximum flux $F_{\rm max}$ as a function of distance $d$ of the sample. 
%%%
\begin{equation}
aiu
\end{equation}
%%%

\vspace{2in}

\begin{thebibliography}{}
\bibitem[Schroder(2007)]{Schroder2007} Schr\"{o}der , K.-P. \& Cuntz, M. 2005, ApJ, 630,  L73 
\bibitem[Schroder(2007)]{Schroder2007} Schr\"{o}der , K.-P. \& Cuntz, M. 2007, A\&A, 465, 593 
\end{thebibliography}

\end{document}


