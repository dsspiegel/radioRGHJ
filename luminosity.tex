%\documentclass[12pt,preprint]{aastex}
\documentclass[12pt]{emulateapj} 
\usepackage{amsmath}
\usepackage{amssymb}
\usepackage{amstext}
%\usepackage{apjfonts}
%\usepackage{graphicx}
\usepackage{epstopdf}
\usepackage{epsfig}
\usepackage{url}
%------------------------------------------
% User-defined macros.
\usepackage{color}
\def\plotoneh#1{\centering \leavevmode
\includegraphics[clip=, width=.95\columnwidth]{#1}}

\allowdisplaybreaks[1]

\definecolor{myColor}{rgb}{0.9,0.9,0.9}  
%------------------------------------------
%%%%%%%%%%%%%%%%%%%%%%%%%%%%%%%%%%%%%%%%%%%%%%
\begin{document}
\renewcommand\bottomfraction{.9}
\newcommand{\PD}[2]{\frac{\partial {#1}}{\partial {#2}}}
\newcommand{\memoYF}[1]{\color{red} #1 \color{black}}

\title{Luminosity function of RGHJs}
\author{Yuka Fujii\altaffilmark{1}}
%\altaffiltext{1}{Institute of Astronomy, Madingley Road, University of Cambridge, Cambridge CB3 0HA, United Kingdom}

\begin{abstract}
N/A
\end{abstract} 

\keywords{planetary systems --- planets and satellites: general}

%%%%%%%%%%%%%%%%%%%%%%%%%%%%%%%%%%%%%%%%%%%%%%%%%%%%%%%
\section{Distribution of observed flux} 
\label{sec:luminosity function}


%In principle, if the expected luminosity function and positional distribution of possible RGHJ emitters were known, one could estimate the likelihood of detecting RGHJs in radio surveys. For illustration we consider a hypothetical power-law distribution for the RGHJ luminosity function $N[L] \propto L^{-\alpha_L}$, where, $N[L]$ is the number of objects per unit volume with a luminosity of $L$, and assume an isotropic distribution of sources in the observed volume. 
%Under these assumptions, the largest luminosity in a volume limited sample (volume $V$, distance $d$) would be such that $V n[L] = 1$, i.e., $L \propto d^{3/\alpha _L}$.
%The flux of such target at distance $d$ is $F \propto d^{\frac{3}{\alpha_L} - 2}$. 
%Therefore,the likelihood of finding the brighter source increases with distance for $\alpha_L < 3/2$. 
%Such a scenario would indeed be favorable to detect RGHJs given that RGs and AGB stars with high mass loss rates are rare and would require probing out to larger distances for detection. 
%For example, for an $N[L] \propto L^{-1}$ the apparently brightest source within a given distance will increase linearly with distance, making it favorable for detecting intrinsically rare objects. 
%Therefore, if we can estimate the luminosity function of RGHJ emitters, we may estimate their detectability.

%Planning future targeted surveys of RGHJs would require accurate distribution functions of various parameters of giant stars in our neighborhood of the Galaxy. 


We consider the number density of observed flux from targets within the distance $R$. 
Luminosity and (observed) flux are denoted by $L$ and $F$, respectively. 
\vspace{1\baselineskip}

Let us denote the number density of evolved stars per unit volume by $n$ and assume that the luminosity function can be written as follows. 
%%%
\begin{equation}
\frac{dn}{dL} = \left\{
\begin{array}{ll}
C L^{-\alpha } & (L_{\rm min} < L < L_{\rm max}) \\
0 & (\mbox{elsewhere})
\end{array}
\right. 
\end{equation}
%%%
Here, $C $ is the proportional coefficient. 

Then, let us denote the (expected) total number of evolved stars in a sphere of radius $R$ by $N_R$. Consider a spherical shell at radius $r$ ($r<R$) with thickness $dr$. The number of evolved stars in this spherical shell and with luminosity ranging from $L\sim L+dL$ is:
%%%
\begin{equation}
\frac{d^2N_R}{dL dr} dL dr = 4 \pi C L^{-\alpha } r^2 dL dr \equiv \Psi (L, r) dL dr \label{eq:d2N/dLdr}
\end{equation}
%%%

The relationship between flux $F$ and luminosity $L$ is, as usual,
%%%
\begin{equation}
F = \frac{L}{4\pi r^2} 
\end{equation}
%%%
where $r$ is the distance to the target. 

Now, we want to get the number of targets with flux between $F$ and $F+dF$. 
To do so, we want to integrate equation (\ref{eq:d2N/dLdr}) over the parameter space $\mathcal{S}$ satisfying
%%%
\begin{equation}
F < \left( \frac{L}{4\pi r^2} \right) < F+dF 
\end{equation}
%%%
To do so, we change the set of variables from $\{ L,\,r \}$ to $\{ F,\,r \}$. 
Thus,
%%%
\begin{eqnarray}
\frac{dN_R}{dF} dF &=& \iint _{\mathcal{S}}  dr dL \;\; \Psi (L, r) \\
&=& \int _0^{R} dr \int _{F}^{F+dF} dF' \;\; \Psi (4\pi r^2F', r)\; det(J) 
\end{eqnarray}
%%%
where
%%%
\begin{equation}
det(J) = \left| \PD{r}{r}\PD{L}{F'} - \PD{r}{F'}\PD{L}{r} \right| = 4 \pi r^2
\end{equation}
%%%

Therefore, 
%%%
\begin{eqnarray}
\frac{dN_R}{dF} dF &=& \left[ \int _0 ^{R} dr \; \Phi (F, r) \right] dF\\
\Phi (F, r) &=& \left\{
\begin{array}{ll}
C' F^{-\alpha } r^{4-2\alpha} & (\sqrt{L_{\rm min}/4 \pi F} < r < \sqrt{L_{\rm max}/4 \pi F}) \\
0 & (\mbox{elsewhere})
\end{array}
\right. \\
C' &\equiv & C ( 4 \pi )^{2-\alpha }
\end{eqnarray}
%%%

Thus, when $\alpha \not = 2.5$,
%%%
\begin{eqnarray}
&& \frac{dN_R}{dF} dF \\
&& = \left\{
\begin{array}{l}
0 \;\;\;\; \displaystyle \left( F < \frac{L_{\rm min}}{4\pi R^2} \right) \\
\\
\displaystyle \frac{C' F^{-\alpha }}{5 - 2\alpha} \left[ R^{5 - 2\alpha} - \left( \sqrt{ \frac{L_{\rm min}}{4\pi F}} \right)^{5 - 2\alpha} \right] dF \\
\\
\;\;\;\;\;\;\;\;\;\;\;\;\;\;  \displaystyle \left( \frac{L_{\rm min}}{4\pi R^2} < F < \frac{L_{\rm max}}{4\pi R^2} \right) \\	
\\
\displaystyle \frac{C' F^{-\alpha }}{5 - 2\alpha} \left[ \left( \sqrt{ \frac{L_{\rm max}}{4\pi F} } \right)^{5 - 2\alpha} - \left( \sqrt{\frac{L_{\rm min}}{4\pi F}} \right)^{5 - 2\alpha } \right] dF \\
\\
\;\;\;\;\;\;\;\;\;\;\;\;\;\; \displaystyle \left( \frac{L_{\rm max}}{4\pi R^2} < F \right) 
\end{array}
\right. \\
%%%
&& = \left\{
\begin{array}{l}
0 \;\;\;\; \displaystyle \left( F < \frac{L_{\rm min}}{4\pi R^2} \right) \\
\\
\displaystyle \frac{C' }{(5 - 2\alpha) (4 \pi )^{2.5-\alpha } } \left[ R^{5 - 2\alpha}F^{-\alpha }  (4 \pi )^{2.5-\alpha }  -  L_{\rm min}^{2.5 - \alpha} F^{-2.5} \right] dF \\
\\
\;\;\;\;\;\;\;\;\;\;\;\;\;\;  \displaystyle \left( \frac{L_{\rm min}}{4\pi R^2} < F < \frac{L_{\rm max}}{4\pi R^2} \right) \\	
\\
\displaystyle \frac{C' F^{-2.5 }}{(5 - 2\alpha) (4 \pi )^{2.5-\alpha } } \left[ L_{\rm max}^{2.5 - \alpha} - L_{\rm min}^{2.5 - \alpha}  \right] dF \\
\\
\;\;\;\;\;\;\;\;\;\;\;\;\;\; \displaystyle \left( \frac{L_{\rm max}}{4\pi R^2} < F \right) 
\end{array}
\right. 
\end{eqnarray}
%%%

\newpage


%%%%%%%%%%%%%%%%%%%%%%%%%%%%%%%%%%%
\begin{figure}[htbp]
   \plotoneh{dndF_alpha31.pdf}
   \plotoneh{dndF_alpha15.pdf}
   \caption{Number density of observed flux in $\mu {\rm Jy}$.}
  \label{fig:dndF}
\end{figure}
%%%%%%%%%%%%%%%%%%%%%%%%%%%%%%%%%%% 



For now, let us suppose
%%%
\begin{itemize}
\item range of luminosity of ``evolved stars'':\\$L_{\odot }-10,000 L_{\odot }$
\item $\Rightarrow$ minimum radio luminosity of ``RGHJs''\\%
$L_{\rm min} = 2,1\times 10^{11}/30 {\rm MHz} = 7000 {\rm W/Hz}$
\item $\Rightarrow $maximum radio luminosity of ``RGHJs'':\\$L_{\rm max} = 4.5\times 10^{13} {\rm W}$
\end{itemize}
%%%
With these assumptions, number density of observed flux is shown in Figure \ref{fig:dndF}. We consider $\alpha = 3.1$ (see the last section) and $\alpha = 1.5$ for demonstration. 

\vspace{2\baselineskip}

\section{Expected value for the highest flux within radius $R$}


Let us consider that the probability of finding an object with flux in the range between $F$ and $F + dF$ is
%%%
\begin{equation}
P (F) dF = \frac{1}{N_R} \frac{dN_R}{dF}
\end{equation}
%%%
Note that $N_R$ is the {\it expected} total number of targets within radius $R$. \memoYF{not sure how to deal with the fact that this is expected number.}

Then, the probability to find one object with flux higher than $\tilde F$ is
%%%
\begin{equation}
P (> \tilde F) = \int _{\tilde F} ^{\infty}  P (F) \; dF 
\end{equation}
%%%
In turn, the probability to find an object with flux lower than $\tilde F$ s
%%%
\begin{equation}
P (< \tilde F) = 1 - P (> \tilde F)
\end{equation}
%%%
Thus, the probability to find the maximum flux between $\tilde F \sim \tilde F + d\tilde F$, $\mathcal{P} (\tilde F) d\tilde F$ is
%%%
\begin{eqnarray}
\mathcal{P} (\tilde F) d\tilde F &=& N P(\tilde F) d\tilde F \cdot \left[ P (< \tilde F) \right]^{N-1} \\
&=& N P(\tilde F) d\tilde F \cdot \left[ 1 - P (> \tilde F) \right]^{N-1} \\
&\sim & N P(\tilde F) d\tilde F \cdot \left[ 1 - ( N -1 ) P (> \tilde F) \right]
\end{eqnarray}
%%%

So the expected value for the maximum flux, $<F_{\rm max}>$ is:
%%%
\begin{eqnarray}
<F_{\rm max}> &=& \int \tilde F \mathcal{P} (\tilde F) \; d\tilde F \\
&=& ...
\end{eqnarray}
%%%


\memoYF{Not sure if the argument above is right}

\vspace{2\baselineskip}

\section{Power index of luminosity function (?)}

In order to estimate the luminosity function of RGHJ emitters, one of the key factors is the distribution of mass-loss rates of giant host stars. 
For a given set of orbital parameters the luminosity ($L$) of a RGHJ depends on the stellar mass-loss rate and wind velocity as $\dot{M}_\star^{2/3} v^{5/3}$. 
Employing equation (\ref{eq:mass-loss}) which implies $\dot{M}_\star \propto L_\star^{7.5/4} R_\star^{-0.75} M_\star^{-1} (1+A/g_\star)$ and crudely estimating the wind velocity as $v \propto \sqrt{GM/R} \propto M^{1/2} R^{-1/2}$, the luminosity of the RGHJ radio emission is given by $L \propto L_\star^{7/12} M_\star^{1/6} T_{\rm eff}^{8/3} (1+A/g_\star)^{2/3}$. 
Given that the dependences on $M_\star^{1/6}$ is relatively weak and that $T_{\rm eff}$ does not change significantly, the radio luminosity of RGHJs scales roughly as $L \propto L_{\star }^{7/12}$. 
Observations of globular clusters indicates the luminosity function of red giants is $dn/dL_{\star } \sim L_{\star }^{-1.8}$ (Sandquist et al. (1996)) %\citep{sandquist1996}
. Therefore, $\alpha \sim -3.1$. 





\end{document}


The maximum flux of volume $4\pi R^3/3$ is likely to occur (?) at $F_{\rm max}$ such that
%%%
\begin{equation}
\int _{F_0}^{\infty} \frac{dn}{dF} dF = 1
\end{equation}
%%%

Solar luminosity becomes 2 times brighter than current value when it evolves off the main-sequence stage. Correspondingly, the radio wave from Jupiter will becomes $2.1 \cdot 10^{11} \times 2^{7/12} = 3.1 \cdot 10^{11}~{\rm W} = 2.5 \cdot 10^{11} ~{\rm W/str}$ . 

\vspace{2\baselineskip}

\begin{itemize}
\item $L_{\rm min} = 2.5 \times 10^{11} ~{\rm W/str}$
\item $L_{\rm max} = $
\end{itemize}

\vspace{2\baselineskip}


\vspace{2\baselineskip}



\end{document}


 
%%%
\begin{eqnarray}
&& \frac{dn}{dF} dF \\
&& = \left\{
\begin{array}{l}
0 \;\;\;\; \displaystyle \left( F < \frac{L_{\rm min}}{R^2} \right) \\
\\
\displaystyle \frac{4\pi C F^{-\alpha }}{5 - 2\alpha} \left[ R^{5 - 2\alpha} - \left( \sqrt{ \frac{L_{\rm min}}{F}} \right)^{5 - 2\alpha} \right] dF \\
\\
\;\;\;\;\;\;\;\;\;\;\;\;\;\;  \displaystyle \left( \frac{L_{\rm min}}{R^2} < F < \frac{L_{\rm max}}{R^2} \right) \\	
\\
\displaystyle \frac{4\pi C F^{-\alpha }}{5 - 2\alpha} \left[ \left( \sqrt{ \frac{L_{\rm max}}{F} } \right)^{5 - 2\alpha} - \left( \sqrt{\frac{L_{\rm min}}{F}} \right)^{5 - 2\alpha } \right] dF \\
\\
\;\;\;\;\;\;\;\;\;\;\;\;\;\; \displaystyle \left( \frac{L_{\rm max}}{R^2} < F \right) 
\end{array}
\right.
\end{eqnarray}
%%%



%%%
\begin{eqnarray}
&&<F> = \int F \frac{dn}{dF} dF \\
%
&&= \frac{4 \pi C}{5 - 2 \alpha } \\
&& \times \left[ \int _{F_{\rm l}} ^{F_{u}} F^{1-\alpha } \left\{ R^{5-2\alpha } - \left( \frac{L_{\rm min}}{F} \right)^{2.5-\alpha} \right\} \right. dF \\
&&+ \left. \int _{F_{u}} ^{\infty } F^{1-\alpha } \left\{ \left( \frac{L_{\rm max}}{F} \right)^{2.5-\alpha} - \left(\frac{L_{\rm min}}{F} \right)^{2.5-\alpha} \right\} \right] dF \\
%
&&= \frac{4 \pi C}{5 - 2 \alpha } \\
&& \times \left\{ \left[ \frac{F^{2-\alpha } R^{5-2\alpha }}{2-\alpha } - \frac{F^{-0.5}L_{\rm min}^{2.5-\alpha} }{-0.5} \right]_{F_l} ^{F_u} \right. \\
&&\left. + \left[ \frac{F^{-0.5}L_{\rm max}^{2.5-\alpha} }{-0.5} - \frac{F^{-0.5}L_{\rm min}^{2.5-\alpha} }{-0.5} \right]_{F_l} ^{F_u} \right\} \\
&&\propto R 
%
\end{eqnarray}
%%%

Therefore, 
%%%
\begin{eqnarray}
&& \frac{dn}{dF} dF \\
&& = \left\{
\begin{array}{l}
0 \;\;\;\; (R < \sqrt{L_{\rm min}/F}) \\
\\
\displaystyle \frac{4\pi C F^{-\alpha }}{5 - 2\alpha} \left[ R^{5 - 2\alpha} - \left( \sqrt{ \frac{L_{\rm min}}{F}} \right)^{5 - 2\alpha} \right] \\
\\
\;\;\;\;\;\;\;\;\;\;\;\;\;\;  (\sqrt{L_{\rm min}/F} < R < \sqrt{L_{\rm max}/F} ) \\
\\
\displaystyle \frac{4\pi C F^{-\alpha }}{5 - 2\alpha} \left[ \left( \sqrt{ \frac{L_{\rm max}}{F} } \right)^{5 - 2\alpha} - \left( \sqrt{\frac{L_{\rm min}}{F}} \right)^{5 - 2\alpha } \right] \\
\\
\;\;\;\;\;\;\;\;\;\;\;\;\;\; (\sqrt{L_{\rm max}/F} < R) 
\end{array}
\right.
\end{eqnarray}
%%%

%%%%%%%%%%%%%%%%%%%%%%%%%%%%%


The probability to have the maximum luminosity of $F \sim F + dF$, $\mathcal{P} (F_{\rm max})$, with $N$ samples is
% $expected value of the maximum luminosity $L_{\rm max}$ is
%%%
\begin{eqnarray}
\mathcal{P}(F_{\rm max}) &\cdot & F_{\rm max} \\
&=& N \cdot p (F_{\rm max}) \cdot P^{N-1}(<F_{\rm max}) \cdot dF_{\rm max}\end{eqnarray}
%%%

%%%%%%%%%%%%%%%%%%%%%%%%%%%%%%%%%%%%%%%%%%%%%%%%%%%%%%%
\section{Luminosity Function Note \#2} 
\label{sec:luminosity function}

Assume 
%%%
\begin{equation}
\frac{dn}{dL} = L^{-\alpha }
\end{equation}
%%%
Probability for a star to have luminosity $L\sim L+dL$ is 
%%%
\begin{equation}
p (L) dL = C L^{-\alpha }
\end{equation}
%%%


Probability for a star to have luminosity larger than $L$ is 
%%%
\begin{equation}
P(>L) = \int _L^{\infty} C L^{-\alpha } dL = C \frac{L^{-\alpha +1}}{-\alpha + 1}
\end{equation}
%%%

Probability for a star to have luminosity less than $L$ is 
%%%
\begin{equation}
P(<L) = 1 -P (>L) \end{equation}
%%% 

The probability to have the maximum luminosity of $L \sim L + dL$, $\mathcal{P(L_{\rm max}})$, with $N$ samples is
% $expected value of the maximum luminosity $L_{\rm max}$ is
%%%
\begin{eqnarray}
\mathcal{P}(L_{\rm max}) &\cdot & dL_{\rm max} \\
&=& N \cdot p (L_{\rm max}) \cdot P^{N-1}(<L_{\rm max}) \cdot dL_{\rm max} \\
&=& N C L_{\rm max}^{-\alpha } \left[ 1 - \left( \frac{CL_{\rm max}^{-\alpha+1}}{-\alpha+1} \right) \right]^{N-1}
\end{eqnarray}
%%%
Thus, the expected value for the maximum luminosity is
%%%
\begin{eqnarray}
< L_{\rm max} > (N) &=& \int L_{\rm max} \cdot \mathcal{P_{\rm max}}(L_{\rm max}) \cdot dL_{\rm max} \\
&=& N C \int dL_{\rm max} \cdot L_{\rm max}^{-\alpha +1} \left[ 1 - \left( \frac{CL_{\rm max}^{-\alpha+1}}{-\alpha+1} \right) \right]^{N-1} \\
&\equiv & N C \int dL_{\rm max} \cdot L_{\rm max}^{-\alpha +1} K^{N-1}(L_{\rm max})
\end{eqnarray}
%%%

%Thus, the expected value for the maximum flux from the spherical shell at $l \sim l+dl$, $f_{\rm max} (l)$, is:
%%%
%\begin{equation}
%f_{\rm max} (l) = < L_{\rm max} > (4\pi l^2 dl) / l^2
%\end{equation}
%%%


As we observer further (i.e. while the distance $l$ increases by d$l$), the maximum luminosity increases
%%%
\begin{eqnarray}
&& \frac{d < L_{\rm max} >}{dl} = \frac{dN}{dl} \frac{d < L_{\rm max} >}{dN} \\
&\propto & l^{2} \left[ \int dL_{\rm max} \cdot L_{\rm max}^{-\alpha +1} K^{N-1}(L_{\rm max}) \right. \\
&&\left. + N (N-1) \int dL_{\rm max} \cdot L_{\rm max}^{-\alpha +1} K^{N-2}(L_{\rm max}) \right]
\end{eqnarray}
%%%



Let us compute the expected value for the maximum flux $F_{\rm max}$ as a function of distance $d$ of the sample. 
%%%
\begin{equation}
aiu
\end{equation}
%%%

\vspace{2in}

\begin{thebibliography}{}
\bibitem[Schroder(2007)]{Schroder2007} Schr\"{o}der , K.-P. \& Cuntz, M. 2005, ApJ, 630,  L73 
\bibitem[Schroder(2007)]{Schroder2007} Schr\"{o}der , K.-P. \& Cuntz, M. 2007, A\&A, 465, 593 
\end{thebibliography}

\end{document}


