\documentclass[iop,numberedappendix,apj]{emulateapj}

\usepackage{epsfig}
\usepackage{amsmath}
\usepackage{rotating}
\usepackage{natbib}
\usepackage{enumerate}
\usepackage{multirow}
\usepackage{array}
\usepackage{appendix}
\usepackage{comment}
\usepackage{color,xcolor}
\usepackage{url}
\usepackage{hyperref}
\hypersetup{colorlinks,linkcolor={blue!50!black},citecolor={blue!50!black},urlcolor={blue!50!black}}
\allowdisplaybreaks[1]

\bibliographystyle{apj}

\definecolor{DarkGreen}{rgb}{0.0, 0.6, 0.0}
\def\memoTM#1{\color{DarkGreen}$[${\bf #1}$]$ \color{black}}

\def\plotonesc#1{\centering \leavevmode
\includegraphics[clip=, width=1.70\columnwidth]{#1}}
\def\plotoneh#1{\centering \leavevmode
\includegraphics[clip=, width=.95\columnwidth]{#1}}
\def\plotone#1{\centering \leavevmode
\includegraphics[clip=, width=.85\columnwidth]{#1}}
\def\plotoneShrinkSmall#1{\centering \leavevmode
\includegraphics[clip=, width=.49\columnwidth]{#1}}
\def\plotoneShrinkMed#1{\centering \leavevmode
\includegraphics[clip=, width=.55\columnwidth]{#1}}
\def\plotoneShrinkBig#1{\centering \leavevmode
\includegraphics[clip=, width=.65\columnwidth]{#1}}
\def\plottwo#1#2{\centering \leavevmode
\includegraphics[width=.45\columnwidth]{#1} \hfil
\includegraphics[width=.45\columnwidth]{#2}}
\def\plottwob#1#2{\centering \leavevmode
\includegraphics[width=.49\columnwidth]{#1} \hfil
\includegraphics[width=.49\columnwidth]{#2}}
\def\plottwor#1#2{\centering \leavevmode
\includegraphics[width=.55\columnwidth,angle=90]{#1} \hfil
\includegraphics[width=.55\columnwidth,angle=90]{#2}}
\def\plotthree#1#2#3{\centering \leavevmode
\includegraphics[width=.3\columnwidth]{#1} \hfil
\includegraphics[width=.3\columnwidth]{#2} \hfil
\includegraphics[width=.3\columnwidth]{#3}}

\def\gsim{\;\rlap{\lower 2.5pt
 \hbox{$\sim$}}\raise 1.5pt\hbox{$>$}\;}
\def\lsim{\;\rlap{\lower 2.5pt
   \hbox{$\sim$}}\raise 1.5pt\hbox{$<$}\;}

\newcommand{\Tcmb}{\mbox{$T_{\mbox{\tiny CMB}}$}}
\newcommand{\Tsky}{\mbox{$T_{\mbox{\tiny sky}}$}}
\newcommand{\Tsys}{\mbox{$T_{\mbox{\tiny sys}}$}}
\newcommand{\Trx}{\mbox{$T_{\mbox{\tiny rx}}$}}
\newcommand{\sigRMS}{\mbox{$\sigma_{\mbox{\tiny RMS}}$}}

% set formatting properties
\setlength{\textwidth}{6.5in}
\setlength{\textheight}{8.8in}
\setlength{\hoffset}{0.0in}
\setlength{\voffset}{-0.4in}
\parindent 0.2in
\parskip 0.1in



%%%%%%%%%%%%%%%%%%%%%%%%%%%%%%%%%%%%%%%%%%%%%%%%%
% THE DOCUMENT BEGINS HERE                      %
%%%%%%%%%%%%%%%%%%%%%%%%%%%%%%%%%%%%%%%%%%%%%%%%%

\slugcomment{Submitted to ApJ, XX September 2015}

\begin{document}

%%% Begin front material
%\twocolumn[%%% Begin front material

\title{Radio Emission from Red-Giant Hot Jupiters}

\author{
%
Yuka Fujii\altaffilmark{1} 
%
David S. Spiegel\altaffilmark{2,3,4}
%
Tony Mroczkowski\altaffilmark{5,6}
%
Jason Nordhaus\altaffilmark{7,8}\\
%
Neil T. Zimmerman\altaffilmark{9}
%
Aaron Parsons\altaffilmark{10}
%
Nikku Madhusudhan\altaffilmark{11}
%
Mehrdad Mirbabayi\altaffilmark{4}
}

\affil{$^1$Earth-Life Science Institute, Tokyo Institute of Technology, 
  Tokyo, JAPAN, 152-8550}
  
\affil{$^2$Analytics \& Algorithms, Stitch Fix,
  San Francisco, CA  94103}

\affil{$^3$Research \& Development, Sum Labs,
  New York, NY  10001}

\affil{$^4$Astrophysics Department, Institute for Advanced Study,
  Princeton, NJ 08540}

\affil{$^5$National Research Council Fellow}

\affil{$^6$Naval Research Laboratory, 4555 Overlook Ave SW, Washington, DC 20375}

\affil{$^7$Department of Science and Mathematics, National Technical Institute for the Deaf, Rochester Institute of Technology, Rochester, NY 14623}

\affil{$^8$Center for Computational Relativity and Gravitation, Rochester Institute of Technology, Rochester, NY 14623}

\affil{$^9$Space Telescope Science Institute, 3700 San Martin Drive, Baltimore, MD 21218}

\affil{$^{10}$Astronomy Department, University of California Berkeley}

\affil{$^{11}$Astronomy Department, University of Cambridge, UK}
\vspace{0.5\baselineskip}

\email{
yuka.fujii@elsi.jp
}


\begin{abstract}
%%%
When planet-hosting stars evolve off the main sequence and go through the red-giant branch, the stars become orders of magnitudes more luminous and at the same time lose mass at much higher rates than their main-sequence counterparts.
Accordingly, planetary companions around them at orbital distances of a few AU, if they exist, will be heated up to the level of canonical hot Jupiters and also subjected to a dense stellar wind.
Given that magnetized planets interacting with stellar winds emit radio waves, such ``Red-Giant Hot Jupiters'' (RGHJs) may also be candidate radio emitters.
We estimate the spectral auroral radio intensity of RGHJs based on the empirical relation with the stellar wind as well as a proposed scaling for planetary magnetic fields. 
We predict that RGHJs might be intrinsically as bright as canonical hot Jupiters, and about 100 times brighter than equivalent objects around main-sequence stars.  
The signal from a RGHJ may be detectable at distances up to a few hundred parsecs with the Square Kilometer Array. 
\end{abstract}


\keywords{planets and satellites: Jupiter --- Sun: evolution ---
  planetary systems --- stars: evolution ---
  stars: AGB and post-AGB --- radio continuum: planetary systems}
  
%]%%% End front material


%%%%%%%%%%%%%%%%%%%%%%%%%%%%%%%%%%%%%%%%%%%%%%%%%%%%%%%%%%%%%%%%%%%
\section{Introduction}
\label{sec:intro}
%%%%%%%%%%%%%%%%%%%%%%%%%%%%%%%%%%%%%%%%%%%%%%%%%%%%%%%%%%%%%%%%%%%


Planets with strong magnetic fields may generate radio and/or X-ray emission when interacting with energetic charged particles. 
It is well known that Jupiter emits radio waves from its auroral region due to the cyclotron-maser instability \citep[e.g.][]{wu1979,zarka1998}.  Potentially, exoplanets can also generate radio emission through similar mechanisms, depending on their intrinsic magnetic fields and the density of surrounding plasmas, e.g. stellar wind particles and particles from Io-like moons. 
Observations of radio emission from Solar System planets imply an empirical relation that the radio intensity is proportional to the input solar wind flux going into each planetary magnetosphere, which is commonly referred to as the ``radiometric Bode's law'' \citep{desch+kaiser1984}. 
Extrapolating this scaling to exoplanets, radio emission from exoplanetary systems has been examined. 
\citet{farrell1999,zarka2001,lazio2004} estimated that a few of the known exoplanetary systems may have radio flux of $\sim $1 mJy level, due to the small orbital distance of the planet and/or the larger planetary mass. 
\citet{stevens2005} gave improved estimates of the stellar mass-loss rate based on X-ray flux and re-evaluated the radio flux of known exoplanetary systems. 
\citet{griesmeier2005} took account of the high stellar activity in the early stage of the planetary system and proposed that the young system would be good candidates to search for radio emission. 
\citet{griesmeier2007a, griesmeier2007b} discussed the effects of the detailed properties of stellar wind in the proximity of the stars. They considered not only the kinetic energy of the stellar wind but also the magnetic energy of the stellar wind and coronal mass ejection (CME). 
Note that in these papers, the scaling relation of planetary magnetic fields differ from paper to paper.
In \citet{reiners2010}, the authors adopted a new scaling relation of planetary magnetic field based on \citet{christensen_et_al2009}. 
Although observational searches for these radio signatures are underway, no clear detection has been claimed \citep{bastian2000,george2007,stroe2012,hallinan2013,murphy2015}, though there are some promising initial results \citep{lecavelier_et_al2013,sirothia2014}.

When stars less than $\sim$8~$M_\sun$ evolve off the main sequence, they evolve through the red-giant branch (RGB) and the asymptotic-giant branch (AGB) phases where their radii and luminosities increase by orders of magnitude.
Jovian planets in orbit around such stars can migrate inward or outward due to the interplay between tidal torques and mass-loss process on the post-main sequence \citep{nordhaus_et_al2010, spiegel2012, mustill+villaver2012, nordhaus+spiegel2013}.
During this time, such planets can be transiently heated to hot-Jupiter temperatures ($\gtrsim$1000~K) at distances out to tens of AU, depending on the star's mass; such planets are termed ``Red Giant Hot Jupiters'' (RGHJs), though the term can refer to planets orbiting either RGB and AGB stars \citep{spiegel+madhusudhan2012}.
Such planets are also subject to interactions with a massive (but slow) stellar wind, as the mass-loss rate of evolved stars is significant, ranging from $\sim 10^{-8} M_\odot$/yr to $\sim  10^{-5} M_\odot$/yr with the highest values for AGB stars \citep[e.g.,][]{reimers1975, schild1989, vassiliadis1993, schoier2001, vanloon2005}. 
On the assumption that the radio emission is correlated with the stellar wind, planetary companions around evolved stars could also generate bright radio emission. 

Based on this speculation, \citet{ignace2010} examined radio emission from known substellar-mass companions around cool evolved stars.
They found that the low ionization fraction of the stellar wind of evolved stars suppresses their radio emission and leads to weak radio emission, probably below the detection limit. 

In this paper, we argue that the accretion of the massive stellar wind onto the planet would emit UV and X-ray photons that ionize the stellar wind in the vicinity of the planet.
The ionized stellar wind particles would then interact with the planetary magnetic field in the same way as Solar System planets do.
Thus, by extrapolating the ``radiometric Bode's law,'' we provide more optimistic estimates for highly evolved stars compared to the previous result.
We also consider the plasma frequency cut-off of the stellar wind, which turns out to be one of the major obstacles to detect radio emission from RGHJs. 
Given that the survey of exoplanets around highly evolved stars is not complete, we do not constrain ourselves to the systems known to have sub-stellar companions.
Instead, we investigate the parameter space where we can search for radio emission and examine the possibility to detect them with current and future instruments. 

In Section \ref{s:assumptions}, we introduce the framework to obtain the frequency of radio emission and the planetary radio emission, and describe our models for the stellar wind and planetary magnetosphere.
Section \ref{s:result} presents an estimate of the spectral radio intensity of RGHJs and compares the predictions with what might be expected from canonical hot Jupiters as well as those from Jupiter-twins.
Section \ref{s:observability} gives the prospects for the signal detection with the current/future instruments. 
Estimates for the actual late-type (M type) evolved stars are also included. 
Section \ref{s:discussion} is devoted to additional discussions, one for the usefulness of the radio emission to characterize the system and the other for the effects of canceling magnetic field which results from the plasmas spiraling in the planetary magnetosphere.  
Finally, Section \ref{s:conc} concludes the paper with a brief summary. 


%%%%%%%%%%%%%%%%%%%%%%%%%%%%%%%%%%%%%%%%%%%%%%%%%%%%%%%%%%%%%%%%%%%
\section{Model}
\label{s:assumptions}
%%%%%%%%%%%%%%%%%%%%%%%%%%%%%%%%%%%%%%%%%%%%%%%%%%%%%%%%%%%%%%%%%%%


In this section, we describe our scheme to estimate radio emission from RGHJs. 
First, we introduce our framework to compute the frequency and intensity of planetary radio emission in Section \ref{ss:model_frequency} and Section \ref{ss:model_intensity}, respectively. 
Then, two ingredients for the framework --- the strength of planetary magnetic field and the properties of the stellar wind --- will be presented in Section \ref{ss:magneticfield} and Section \ref{ss:stellarwind}, respectively. 
In Section \ref{ss:ionization}, we consider the ionization around the planets, which is a crucial factor to determine the efficiency of the planetary magnetosphere and the stellar wind. 


%%%%%%%%%%%%%%%%%%%%%%%%%%%%%%%%%%%%%%%%%%%%%%%%%%%%%%%%%%%%%%%%%%%
\subsection{Frequency of radio emission}
\label{ss:model_frequency}
%%%%%%%%%%%%%%%%%%%%%%%%%%%%%%%%%%%%%%%%%%%%%%%%%%%%%%%%%%%%%%%%%%%

Planetary auroral radio wave are emitted at the local cyclotron frequency along the motion of the plasmas.
The upper limit is around the cyclotron frequency of the planetary surface magnetic field, $\nu_{\rm cyc,\,max}$: 
%%%%%%%%%% 
\begin{equation}
\nu_{\rm cyc,\,max} = \frac{eB}{2\pi m_e c} \approx 28 {\rm~MHz} \left( \frac{B}{10 \rm~G} \right) \label{eq:fcyc}
\end{equation}
%%%%%%%%%%
where $B$ is the strength of magnetic field at the planetary surface, $e$ and $m_e$ are the charge and mass of the electron, respectively, and $c$ is the speed of light. 

Radio emission from the planet is observable from ground only when the frequencies of the radio emission are larger than the plasma frequency of Earth's ionosphere $\nu_{\rm plasma}^\oplus$ and the maximum plasma frequency along the line of sight $\nu_{\rm plasma}^{\rm los}$: 
%%%%%%%%%% 
\begin{equation}
\nu_{\rm cyc,\,max} > \nu_{\rm plasma}^\oplus \;\;\; \mbox{and} \;\;\; \nu_{\rm cyc,\,max} > \nu_{\rm plasma}^{\rm los}
\end{equation}
%%%%%%%%%%
The plasma frequency may be expressed as
\begin{eqnarray}
\nu_{\rm plasma} & = & \sqrt{\frac{n_e e^2}{\pi m_e}} \\
 & = & 8979 {\rm~Hz} \times \left( \frac{n_e}{\rm cm^{-3}} \right)^{1/2} \, .
\label{eq:fplasma}
\end{eqnarray}
In the Earth's ionosphere, the electron number density is less than $10^6$~cm$^{-3}$, which implies that $\nu_{\rm plasma}^\oplus \lsim 10$~MHz. 
On the other hand, the $\nu_{\rm plasma}^{\rm los}$ depends on the density of stellar wind particles around the planet, which will be specified in Section \ref{ss:stellarwind} below.  


%%%%%%%%%%%%%%%%%%%%%%%%%%%%%%%%%%%%%%%%%%%%%%%%%%%%%%%%%%%%%%%%%%%
\subsection{Flux of radio emission}
\label{ss:model_intensity}
%%%%%%%%%%%%%%%%%%%%%%%%%%%%%%%%%%%%%%%%%%%%%%%%%%%%%%%%%%%%%%%%%%%

The auroral radio spectral flux of exoplanets observed at the Earth, $F_{\nu}$, can be expressed by:
%%%
\begin{equation}
F_{\nu} = \frac{P_{\rm radio}}{\Omega l^2 \Delta \nu}
\label{eq:Fnu}
\end{equation}
%%%
where $P_{\rm radio}$ is the energy that is deposited as radio emission of considered frequency range, $\Omega$ is the solid angle of the emission, $l$ is the distance between the target and the Earth, and $\Delta \nu$ is frequency bandwidth. 

We estimate the radio emission of exoplanets, $P_{\rm radio}$, simply by scaling the Jovian auroral radio emission, $P_{\rm radio,\,J}$, with the input energy from stellar wind, in the same manner as \citet{griesmeier2005,griesmeier2007a,griesmeier2007b}.
The scaling is based on the empirical/apparent good correlation between the radio emission intensity of Solar System planets and the input kinetic energy, $P_{\rm inp,\,k}$, or the magnetic energy of the solar wind, $P_{\rm inp,\,m}$, \citep[``radiometric Bode's law''; ][]{desch+kaiser1984}, i.e.,
%%%
\begin{eqnarray}
P_{\rm radio} &\propto & P_{\rm inp} \\
P_{\rm inp,\,k} &=& n v^3 r_{\rm mag} ^2 \label{eq:Pinp_kin}, \\
P_{\rm inp,\,m} &=& v B_{\star\bot }^2 r_{\rm mag} ^2 \label{eq:Pinp_mag} \, ,
\end{eqnarray}
%%%
where $n$ is the number density of the stellar wind, $v$ is the relative velocity of the stellar wind particles and the planet, $ B_{\star\bot }$ is the interstellar magnetic field perpendicular to the stellar wind flow, and $r_{\rm mag}$ is the radius of the magnetic stand-off point.  
It is not clear from the observations of the Solar System planets which is the fundamental one, the correlation to the input kinetic energy, that to the magnetic energy, or both. 
In this paper, we assume that the radio emission is scaled with the input kinetic energy, i.e., $P_{\rm radio} \propto P_{\rm k,\,inp}$ and consider the possible effects of massive stellar wind. 

In the case that the correlation with input magnetic energy is fundamental, it is difficult to estimate the planetary radio emission at this point because magnetic field of evolved stars are not well constrained for highly evolved stars; most of observations only set an upper limit \citep[e.g.,][]{konstantinova2010,petit2013,tsvetkova2013,konstantinova2013,auriere2015}. 
In the case of M-type giant EK boo, surface magnetic field $\sim 0.1-10 $G has been measured; In this particular case, given the large stellar radius $R_\star \sim 210 R_{\odot }$ the magnetic moment may be about $10^3$ times larger than the Sun and thus may also increase the planetary radio emission.
However, because of the lack of universal understanding of the stellar magnetic field, we leave the magnetic model of radio emission for evolved stars for future work. 

In reality, the total power of Jovian auroral emission varies greatly over time: $1.3\times 10^{10}$ W for the average, $3.2\times 10^{10}$ W for the average of highly active time, and $4.5 \times 10^{11}$ W for the peak activity \citep{zarka_et_al2004}. 
Here we employ $P_{\rm radio,\,J}=2.1\times 10^{11}\mbox{~W}$ as a canonical value, following \citet{griesmeier2005,griesmeier2007b}. 

The radius of the magnetic stand-off point, $r_{\rm mag}$, is obtained based on the balance between the stellar wind pressure and the planetary magnetic pressure: 
%%%
\begin{equation}
m_p n v ^2 \sim \frac{B^2}{8\pi}\left( \frac{r_{\rm mag}}{R_p} \right)^{-6}  \label{eq:stand-off} ,
\end{equation}
%%%
where $m_p$ is the proton mass. Thus, 
%%%
\begin{equation}
r_{\rm mag} \sim R_p \left( 8\pi m_p n \right) ^{-1/6} v^{-1/3} B^{-1/3} \label{eq:stand-off-radius} 
\end{equation}
%%%
The radius obtained for Jupiter from this equality is about half of the actual magnetospheric radius \citep[][]{griesmeier2005}. 
To estimate $r_{\rm mag}$ for RGHJs, we scale this radius according to the parameter dependence of equation~(\ref{eq:stand-off-radius}). 


We assume that the solid angle of the emission, $\Omega $, is same as the Jupiter's one for all of the exoplanets.
In reality, the solid angles of auroral radio emission from  Jupiter, Saturn, and Earth are $\sim 1.6$, $\sim $6.3, and $\sim $3.5, respectively \citep{desch+kaiser1984}, which are on the same order and will not significantly affect our order-of-magnitude estimate of radio emission. 
The bandwidth, $\Delta \nu$, is assumed to be proportional to the representative frequency of the emission, which is the cyclotron frequency, following \citet{griesmeier2007b}.


%%%%%%%%%%%%%%%%%%%%%%%%%%%%%%%%%%%%%%%%%%%%%%%%%%%%%%%%%%%%%%%%%%%
\subsection{Assumptions for planetary magnetic field}
\label{ss:magneticfield}
%%%%%%%%%%%%%%%%%%%%%%%%%%%%%%%%%%%%%%%%%%%%%%%%%%%%%%%%%%%%%%%%%%%

In order to obtain the frequency and the intensity of the radio emission, we need to compute the strength of the magnetic field at the planetary surface, $B$. 
We do so simply by scaling the Jovian magnetosphere at the surface $B_J \sim 10$ G according to the planetary mass and age, based on the predicted scaling relation described below. 

So far, several scaling relations for the planetary magnetic field strength  
have been proposed \citep[e.g.][]{russel1978,busse1976,stevenson1979,mizutani1992,sano1993,starchenko2002,christensen2006,christensen_et_al2009}  
which are summarized and compared with numerical simulations in \citet{christensen2010}. 
We employ the scaling law proposed in \citet{christensen_et_al2009}, which has already been used in \citet{reiners2010} to explore the evolution of planetary magnetic fields.
This scaling law is based on the assumption that the ohmic dissipation energy is a fraction of the available convected energy and was found to be in good agreement with the numerical experiments over a wide parameter space and with known objects from the Earth to stars:
%%%%%%%%%%
\begin{equation}
B_{\rm dynamo} ^2 \propto f_{\rm ohm}\;\rho_{\rm dynamo}^{1/3} \;  (Fq_o)^{2/3} \label{eq:Bscaling} 
\end{equation}
%%%%%%%%%%
where $B_{\rm dynamo}$ is the mean magnetic field in the dynamo region, $f_{\rm ohm}$ is the ratio of ohmic dissipation to the total dissipation, $\rho_{\rm dynamo} $ is the mean density in the dynamo region, $F$ is an efficiency factor of order unity, and $q_o$ is the convected flux at the outer boundary of the dynamo region \citep[see][for the comprehensive description]{christensen_et_al2009}. 
Here, $f_{\rm ohm}$ and $F$ are assumed to be constant for the bodies considered in this paper. 
Dipole magnetic field strength at the planetary surface, denoted by $B$, is then scaled by
%%%%%%%%%%
\begin{equation}
B \propto B_{\rm dynamo} \left( \frac{r_{\rm dynamo}}{R_p} \right)^3.  \label{eq:Bscaling2}
\end{equation}
%%%%%%%%%%
where $r_{\rm dynamo}$ is the radius of the outer boundary of the dynamo region. 

Note that the scaling law, Equation (\ref{eq:Bscaling}), would be reasonable only for rapidly rotating objects.
Indeed, unlike canonical hot Jupiters, RGHJs are not tidal locked to their host stars, so we do not have to assume slow spin rotation rates \citep{spiegel+madhusudhan2012}.
Therefore, we implicitly assume that the RGHJs considered in this paper are rapidly rotating so that they generate planetary magnetic fields through the same mechanism as Jupiter. 


In order to evaluate $\rho _{\rm dynamo}$ and $q_o$, we need models of the internal planetary structure. 
We consider Jupiter-like gaseous planets and assume that the planetary radius is constant at $R_p = R_{p,{\rm J}}$, as numerical calculations show that the radii of gaseous planets over the range of $0.1 M_{p,{\rm J}} < M_p < 10M_{p, {\rm J}}$ (with core mass less than 10\%) converge to $0.8 R_{p,{\rm J}} < R_p < 1.2R_{p, {\rm J}}$ within 1 Gyr \citep{fortney2007}. 
For the density profile, we assume a polytrope gas sphere with index $n=1$, which results in:
%%%%%%%%%% 
\begin{equation}
\rho [r] = \left( \frac{\pi M_p}{4 R_p^3} \right) \frac{\sin \left[ \pi \frac{r}{R_p} \right]}{\left( \pi \frac{r}{R_p} \right)}. \label{eq:rho_r}
\end{equation}
%%%%%%%%%%
We determine the radius of outer boundary of the dynamo region, $r_{\rm dynamo}$, by assuming that the hydrogen becomes metallic when $\rho (r)$ exceeds the critical density $\rho_{\rm crit}=0.7\,\mbox{g/cm}^3$ \citep{exoplanets2006, griesmeier2007b}.
The density of the metallic core, $\rho _{\rm dynamo}$ is obtained by averaging the density in the core. 
In the case of Jupiter, $r_{\rm dynamo, J} = 0.85 R_{\rm J}$ and $\rho_{\rm dynamo, J} = 1.899~{\rm g/cm}^3$.

The scaling of convected heat flux at the outer boundary, $q_o$, is obtained by dividing the age-dependent net planetary luminosity, $L_p$, by the surface area of the core region, i.e., $4\pi r_{\rm dynamo}^2$. 
The time-dependent luminosity is taken from equation (1) of \citet{burrows_et_al2001} \citep[see also][]{marley2007}. 
Ignoring the relatively weak dependence on average atmospheric Rosseland mean opacity leads to:
%%%%%%%%%%
\begin{equation}	
L_p \sim 6.3\times10^{23} \; {\rm erg} \; \left( \frac{t}{4.5 \rm~Gyr} \right)^{-1.3} \left( \frac{M_p}{M_{\rm J}} \right)^{2.64} \, 
\label{eq:burrowsLum}
\end{equation}
%%%%%%%%%%
Therefore, we have
%%%%%%%%%%
\begin{equation}
q_o \sim q_{o, {\rm J}} \left( \frac{t}{4.5 \rm~Gyr} \right)^{-1.3} \left( \frac{M_p}{M_{\rm J}} \right)^{2.64} \left( \frac{r_{\rm dynamo}}{r_{\rm dynamo,\,J}} \right)^{-2}\, .
\label{eq:burrowsHeatFlux}
\end{equation}
%%%%%%%%%%



%%%%%%%%%%%%%%%%%%%%%%%%%%%%%%%%%%%%%%%%%%%%%%%%%%%%%%%%%%%%%%%%%%%
\subsection{Assumptions for stellar wind}
\label{ss:stellarwind}
%%%%%%%%%%%%%%%%%%%%%%%%%%%%%%%%%%%%%%%%%%%%%%%%%%%%%%%%%%%%%%%%%%%

Another ingredient for radio emission is the property of the stellar wind.  
The number density of particles in the stellar wind, $n$, can be expressed as
%%%%%%%%%% 
\begin{equation}
n = \frac{\dot M_\star}{4\pi a^2 m_p v_{\rm sw}}
\label{eq:n}
\end{equation}
%%%%%%%%%% 
where $\dot M_\star$ is the stellar mass loss rate, $a$ is the orbital distance from the star and $m_p$ is proton mass, and $v_{\rm sw}$ is the velocity of the stellar wind.
For the solar wind, $\dot M_\odot \sim 2\times 10^{-14} M_{\odot}$/yr and $v_{\rm sw} \sim 400~{\rm km/s}$ \citep[e.g.,][]{hundhausen1997}.

The mass-loss rate of red giants is typically $\dot M_\star \sim 10^{-8}-10^{-7} M_{\odot}$/yr \citep{reimers1975}, and the rate can be as high as $10^{-5} M_\odot{\rm /yr}$ during the AGB phase \citep{schild1989, vassiliadis1993, schoier2001, vanloon2005}.
Therefore, we have
%%%%%%%%%% 
\begin{equation}
\frac{\dot M_\star}{\dot M_{\odot}} \sim 10^6 - 10^9 \, . \label{eq:scale_Mdot}
\end{equation}
%%%%%%%%%% 

The stellar wind velocity, $v_{\rm sw}$, becomes smaller as the star evolves.
The wind velocity is typically of order the escape velocity at a distance of several stellar radius \citep{suzuki2007}, i.e., $\sim \sqrt{2GM_{\star}/R'_{\star}}$ where $R'_{\star} \sim (4-5) R_{\star}$.
For a star with radius $R_{\star }=100~R_{\odot }$, this results in $v_{\rm sw}\sim 30~{\rm km/s}$.
Therefore, a Solar-mass red giant with radius $R_{\star}=100R_{\odot}$ encounters a stellar wind that is slower by at least an order of magnitude than that of the Sun, $v_{\odot}$. 

Based on equations (\ref{eq:scale_Mdot}), the number density of the stellar wind (equation (\ref{eq:n})) is normalized as follows:
%%%%%%%%%% 
\begin{eqnarray}
n &=& 1.8 \times 10^6 \; {\rm cm^{-3}} \times \left( \frac{a}{5 \; {\rm AU}} \right)^{-2} \notag \\
&&\times \left( \frac{\dot M_\star}{10^{-8} M_{\odot }{\rm /yr}} \right) \left( \frac{v_{\rm sw}}{30~\mbox{km/s}} \right)^{-1}. \label{eq:n_normalized}
\end{eqnarray}
%%%%%%%%%% 

The contributions for the velocity term in equation (\ref{eq:Pinp_kin}), which should be the relative velocity between the planet and the solar wind, are given by the stellar wind velocity, the planetary orbital velocity, and the infall velocity (i.e., the velocity due to falling in the planet's gravitional field).
While $v_{\rm sw}$ is $\sim$30~km/s, the orbital velocity is $10-30~{\rm km/s}$ depending on the orbital distance (ranging from $1-5~{\rm AU}$), and the infall velocity from the planetary gravity is $\sim 10-25~{\rm km/s}$, depending on the planetary mass (ranging from $1 M_{\rm J}-10 M_{\rm J}$). 
In the next section, we simply consider
%%%%%%%%%% 
\begin{equation}
\frac{v}{v_{\odot}} \sim 10^{-1} \, . \label{eq:scale_v}
\end{equation}
%%%%%%%%%%
as a fiducial value for normalization. 


%%%%%%%%%%%%%%%%%%%%%%%%%%%%%%%%%%%%%%%%%%%%%%%%%%%%%%%%%%%%%%%%%%%
\subsection{Ionization of stellar wind particles around the planet}
\label{ss:ionization}
%%%%%%%%%%%%%%%%%%%%%%%%%%%%%%%%%%%%%%%%%%%%%%%%%%%%%%%%%%%%%%%%%%%

As discussed in \citet{ignace2010}, as the stars evolve, the ionization fraction of the stellar wind are lowered down to order of $\sim 10^{-3}$ \citep{drake1987}. Since only charged particles interact with planetary magnetic field, this suggests the inefficient interaction with planetary magnetosphere and hence low input energy for radio emission. 
However, at this stage, the velocity of stellar wind becomes slower than the escape velocity of planetary companion and hence the stellar wind particles are expected to accrete onto the planets. 
\citet{spiegel+madhusudhan2012} computed the accretion luminosity $L_{\rm acc}$ and the temperature $T_{\rm acc}$ as follows: 
%%%%%%%%%
\begin{eqnarray}
L_{\rm acc} \sim &&  10^{25} \; {\rm erg\;s}^{-1} \left( \frac{\dot M_*}{10^{-8} M_{\odot }/{\rm yr}} \right)  \notag \\
&& \times \left( \frac{M_p}{M_{\rm J}} \right)^3 \left( \frac{M_*}{M_{\odot }} \right). \\
%
T_{\rm acc} \sim && 2 \times 10^5 ~\mbox{K} \left( \frac{M_p}{M_{\rm J}} \right). \label{eq:Tacc}
\end{eqnarray}
%%%%%%%%%
The accretion onto planets, therefore, leads to emission of UV/X-ray radiation 
whose characteristic energy $\sim k_B T_{\rm acc} $ exceeds the ionization energy of Hydrogen, $E_{\rm Rydberg} = 13.6$~eV, or $\lambda = 91.2\rm~nm$. 
Therefore, UV/X-ray radiation from accretion will create local ionized region around the planet. 

Let us consider the ionization profile around the planet. 
We suppose a state where the ionization and recombination rates are in equilibrium. 
Denoting the ionization fraction by $x$, the equilibrium state at a distance $r$ from the planet may be approximately represented by 
%%%%%%%%%
\begin{equation}
\frac{\dot N_X}{ 4 \pi r^2 } e^{- \tau } n ( 1-x ) \sigma _H [E_{\rm photon}] = (n x )^2 \beta [T_e] \label{eq:equilibrium} 
\end{equation}
%%%%%%%%%
\begin{equation}
\tau = \int _{R_{\rm J}}^r n(1-x) \sigma_H [E_{\rm photon}] dr 
\end{equation}
%%%%%%%%%
where $\dot N_X$ is the source rate of the photons that can ionize the hydrogen, 
$\sigma _H$ is the cross section of H atoms for X-ray \citep{verner1996}: 
%%%%%%%%%
\begin{equation}
\sigma _H [E_{\rm photon}] \sim 6.3 \times 10^{-18} \;{\rm cm}^2 \cdot \left( \frac{E_{\rm photon}}{E_{\rm Rydberg}} \right)^{-3} \label{eq:sigma_H}
\end{equation}
%%%%%%%%%
where $E_{\rm photon}$ is the energy per photon, 
and $\beta [T_e]$ is the ``class B'' recombination coefficient which is 
$ \beta \sim 2.6 \times 10^{-13}\;{\rm cm^3/sec} $ \citep{pequignot1991} 
at electron temperature of stellar wind of evolved stars, $T_e \sim 10^4 $K \citep{suzuki2007}. 


The source rate is obtained by counting the number of photons whose energy exceeds the ionization energy of Hydrogen ($E_{\rm Rydberg} = 13.6$~eV, 
or $\lambda = 91.2\rm~nm$), which is approximately given by dividing the X-ray accretion luminosity by the characteristic photon energy produced:
%%%%%%%%%
\begin{equation}
\dot{N}_X \sim \frac{L_{\rm acc}}{k_B T_{\rm acc}} \, ,
\end{equation}
%%%%%%%%%
where $k_B$ is Boltzmann's constant.
As a result, 
%%%%%%%%%
\begin{eqnarray}
\dot{N}_X \sim  
  \left\{
    \begin{array}{ll}
      2 \times 10^{35} \;  {\rm sec^{-1}} & \;\;\;(M_p = M_{\rm J}) \\
      3 \times 10^{37} \; {\rm sec^{-1}} & \;\;\;(M_p = 10 M_{\rm J})
    \end{array}. 
  \right.
\end{eqnarray}
%%%%%%%%%

For the accretion onto large planets, however, the characteristic photon energy is so high (equation (\ref{eq:Tacc})) that the released energetic electrons may also ionize other atoms in the vicinity.
The cross section\footnote{This is close to the geometric cross section of the Bohr radius.} of hydrogen for electrons is $\sim 4\times 10^{-17} {\rm cm}^2$ \citep{fite1958}, which implies a mean free path for ionized electrons of $\sim$2$R_J$ in the surrounding medium.
As a result, nearly 2/3 of released energetic electrons will ionize a hydrogen atom within $\sim$2$R_J$, and more than 99\% will ionize a hydrogen atom within $\sim$10$R_J$.

In principle, photons with energy $E_{\rm photon}$ have the potential to ionize $E_{\rm photon}/E_{\rm Rydberg}$ times.
To take this into account, we consider two limiting possibilities:
After an ionizing collision, the energy can (\emph{i}) be split evenly between the two electrons, or it can (\emph{ii}) go entirely into the kinetic energy of one electron and not at all into that of the other (of course, any split in between these extremes is possible, too).
Note that conservation of momentum and energy imply that the proton will not acquire a significant fraction of the energy of the collision.\footnote{Were it otherwise, we'd have to take into account what fraction of a rubber ball's kinetic energy is imparted to the kinetic energy of the Earth when bouncing a ball.}  

Scenario (\emph{i}) implies a cascade where an electron with energy $E_{\rm in}$ ionizes an atom, producing two electrons (an ionizing electron plus the released electron) with energy $ ( E_{\rm in} - E_{\rm Rydberg} ) /2 $ for each.
For example, in an idealized case with $M_p = 10 M_{\rm J}$, $k_B T_{\rm acc} \sim 172\,{\rm eV} \sim 12 \, E_{\rm Rydberg}$, a photoionization could produce an electron with energy of $(12-1) = 11 E_{\rm Rydberg} $, then a second ionization by that electron would produce two photons with energy of $(11-1)/2 = 5 E_{\rm Rydberg}$, and the third ionization by these two photons would produce four photons with energy of $(5-1)/2 = 2 E_{\rm Rydberg}$, etc.
The cascade can proceed to the fourth order at the maximum.

Under scenario (\emph{ii}), the cascade proceeds with the initial $E_{\rm in}$ electron leading to an electron with kinetic energy $E_{\rm in} - E_{\rm Rydberg}$ and another electron with 0 kinetic energy.
Clearly, this cascade can produce a maximum total of $E_{\rm in} / E_{\rm Rydberg}$ free electrons.

In reality, not all released electrons may be able to proceed the next ionization.
If $\xi$ represents the fraction of released electrons that proceed to the next ionization, then the number of ionized atoms $N_i$ released through this cascade (\emph{i}), is
%%%%%%%%%
\begin{eqnarray}
  \nonumber N_i & = & (1-\xi) + 2\xi (1-\xi) + 4 \xi^2 (1-\xi) + 8 \xi^3 \\
  \label{eq:N_i1} & = & 1 + \xi + 2 \xi^2 + 4 \xi ^3 \, .
\end{eqnarray}
%%%%%%%%%
Alternatively, cascade (\emph{ii}) leads to
%%%%%%%%%
\begin{eqnarray}
  \nonumber N_i & = & \frac{1 - \xi^k}{1 - \xi} \\
  \label{eq:N_i2}  & = & 1 + \xi + \xi^2 + \cdots + \xi^{k - 1} \, ,
\end{eqnarray}
%%%%%%%%%
where $k \equiv E_{\rm in} / E_{\rm Rydberg}$ is the maximum number of ionizations for the given initial electron energy.
This limit leads to a value for $N_i$ that is not dramatically different from that of limit (\emph{ii}).
In Appendix \ref{sec:AppendixA}, we estimate the Bhabha(/M{\o}ller) scattering cross section and show that cascade (\emph{ii}) --- unequal recoil energies --- is probably more realistic.

Ultimately, $\dot{N}_X $ in equation (\ref{eq:equilibrium}) is replaced by 
%%%%%%%%%
\begin{equation}
\dot{N}_X \rightarrow N_i \dot{N}_X \, .
\end{equation}
%%%%%%%%%
For a $10 M_{\rm J}$ planet, $N_i$ is probably approximately in the range 5---10.


%%%%%%%%%%%%%%%%%%%%%%%%%%%%%%%%%%%
\begin{figure}[htbp]
   \plotoneh{ionizationfraction.pdf}
   \caption{Profile of ionization fraction due to X-ray from planetary accretion, measured from the surface of the planet. Solid lines show the solutions without the correction for the secondary ionization by ionized electrons, while the dashed lines show the solutions which take the correction into account with efficiency factor $N_i=6$. The vertical arrows show the Str\"omgren radius estimated simply  by equation (\ref{eq:stromgren}). The dotted vertical lines indicate the location of the magnetic stand-off radii, $r_{\rm mag}$,  obtained by equation (\ref{eq:stand-off-radius_RGHJ}) below. }
  \label{fig:ionizationfraction}
\end{figure}
%%%%%%%%%%%%%%%%%%%%%%%%%%%%%%%%%%% 

The ionization fraction $x$ as a solution of equation (\ref{eq:equilibrium}) is shown in Figure \ref{fig:ionizationfraction}. 
When, $\tau \sim n \sigma _H r$ is much smaller than unity and thus $e^{-\tau }$ term can be ignored (that is the case for $M_p = 10 M_{\rm J}$),  the solution is simply
%%%%%%%%%
\begin{eqnarray}
x[r] &=& \frac{-1 + \sqrt{1+4C[r]}}{2C[r]} \\
C[r] &\equiv &   \frac{4 \pi n \beta [T_e] r^2}{\dot{N} \sigma_H[E_{\rm photon}]}  \end{eqnarray}
%%%%%%%%%
The solid lines show the ionization fraction corresponding to no additional ionization by electrons (i.e., $\xi=0$, or $N_i = 1$) and dashed lines show that corresponding to $N_i=6$.
In the figure, the vertical lines show the magnetic stand-off radius obtained by equation (\ref{eq:stand-off-radius_RGHJ}) below.
While the photon rate is large for larger planets, the strong dependence of cross section on the photon energy (equation \ref{eq:sigma_H}) leads to the decrease in the radius of ionized region.
Nevertheless, substantial amount of ionized plasmas are expected around the magnetic stand-off radius.


The extent of ionized region may also be roughly estimated by Str\"omgren radius \citep{stromgren1939}: 
%%%%%%%%%
\begin{equation}
r_{\rm stromgren} = \left( \frac{3}{4\pi} \frac{\dot N_X}{n^2 \beta } \right)^{1/3}
\end{equation}
%%%%%%%%%
which gives
%%%%%%%%%
\begin{eqnarray}
r_{\rm stromgren} \sim  
  \left\{
    \begin{array}{ll}
      67 ~R_J & \;\;\;(M_p = M_{\rm J}) \\
      310 ~R_J & \;\;\;(M_p = 10 M_{\rm J})
    \end{array}
  \right. \label{eq:stromgren}
\end{eqnarray}
%%%%%%%%%
for a planet around a red giants at 5~AU. 
The Str\"omgren radius for $1M_{\rm J} $ and for $10M_{\rm J}$ are also indicated in Figure \ref{fig:ionizationfraction}.
Note, however, that unlike the frequently-demonstrated photoionization problem around O/B-type stars, in the situation of our interest, the Str\"omgren radius is not a sharp edge of ionization anymore, due to smaller source rate and smaller cross section (in this case parameter ``$a$'' in equation (13) of \citealt{stromgren1939} is not small) on account of X-ray photons interacting more weakly with neutral hydrogen than UV photons near the ionization limit.
The Str\"omgren radii are indicated by vertical arrows in Figure \ref{fig:ionizationfraction}. 


As a whole, we expect for the X-ray emission from the accretion onto the planet to ionize the incoming stellar wind.
This implies that the stellar wind particles represented by (equation~\ref{eq:n}) are nearly fully ionized plasmas like Solar wind, and that the stellar wind can interact with the planetary magnetic fields in a rather similar way to the case of Solar System planets.



%%%%%%%%%%%%%%%%%%%%%%%%%%%%%%%%%%%%%%%%%%%%%%%%%%%%%%%%%%%%%%%%%%%
\section{Results}
\label{s:result}
%%%%%%%%%%%%%%%%%%%%%%%%%%%%%%%%%%%%%%%%%%%%%%%%%%%%%%%%%%%%%%%%%%%

%%%%%%%%%%%%%%%%%%%%%%%%%%%%%%%%%%%%%%%%%%%%%%%%%%%%%%%%%%%%%%%%%%%
\subsection{Planetary magnetic field and frequency of radio emission}
\label{ss:Bplanet}
%%%%%%%%%%%%%%%%%%%%%%%%%%%%%%%%%%%%%%%%%%%%%%%%%%%%%%%%%%%%%%%%%%%s



%%%%%%%%%%%%%%%%%%%%%%%%%%%%%%%%%%%
\begin{figure}[bhtp]
   \plotoneh{rho_r_dynamo.pdf}
   \plotoneh{qBf_Christensen.pdf}
   \caption{Upper panels: Radius and average density of the dynamo region as a function of planetary mass. Lower panels: Planetary magnetic field as a function of age for different planetary mass, based on the scaling of magnetic field of \citet{christensen2010} and the evolution of luminosity of \citet{burrows_et_al2001}.} 
  \label{fig:planetaryB}
\end{figure}
%%%%%%%%%%%%%%%%%%%%%%%%%%%%%%%%%%% 

Figure \ref{fig:planetaryB} shows the computed radius ($r_c$) and average density ($\rho_c$) of the dynamo region as function of planetary mass, as well as the heat flux at the outer boundary of the core, the estimated strength of planetary magnetic field $B$ and the corresponding cyclotron frequency ($\nu_{\rm cyc}$) as functions of planetary mass and the age. 
Substituting equation (\ref{eq:burrowsHeatFlux}) into equation (\ref{eq:Bscaling2}) and given that $r_{\rm dynamo} $ does not change significantly, the magnetic field is approximately:
%%%%%%%%%% 
\begin{equation}
B   \sim   10~{\rm G} \left( \frac{M_p }{M_{p,{\rm J}}} \right)^{1.04} \left( \frac{t}{4.5~\rm{Gyr}} \right)^{-0.43} \label{eq:scalingB}
\end{equation}
%%%%%%%%%%
under this model. 
Reasonably, the resultant values agree with \citet{reiners2010}, who adopted the same scaling law for planetary magnetic field; we show this figure just for completeness. 
Note that the cyclotron frequency of Jovian planets typically falls between 10~MHz and 1~GHz. 
In this regime, there are a number of current and near-future radio wave detectors including GMRT, 
LOFAR, HERA, SKA, and potential upgrades to the VLA. 

Since $\nu_{\rm cyc,\,max} > \nu_{\rm plasma}^\oplus $, the radio emission will not be hindered by Earth's ionosphere cut-off. 
On the other hand, it may experience opacity due to the plasma of the stellar wind particles around the planet.
The maximum plasma frequency along the line of sight, $\nu_{\rm plasma}^{\rm los}$, corresponds to that in the vicinity of the planet, if the planet is on the near side of its star to the Earth. Therefore, substituting equation~(\ref{eq:n_normalized}) to $n_e$ in equation~(\ref{eq:fplasma}), 
%%%
\begin{eqnarray}
\label{eq:fplasmalos} \nu_{\rm plasma}^{\rm los} & \sim & 
8979 {\rm~kHz} \times \left( \frac{\dot M_\star}{4 \pi a^2 m_p v} \times 1\rm~cm^3 \right)^{1/2} \\
\label{eq:fplasmalos_scaled} &=& 12 {\rm~MHz} \times \left( \frac{a}{5~\mbox{AU}}\right)^{-1} \left(\frac{v}{30~\mbox{km/s}}  \right)^{-1/2} \notag \\
 & & \times \left( \frac{\dot M_\star}{10^{-8} M_{\odot }{\rm /yr}}\right)^{1/2} \, .
\end{eqnarray}
We can see emission only from where $\nu_{\rm cyc,\,max} > \nu_{\rm plasma}^{\rm los}$. Therefore, for Jupiter-mass planets, only distant ones at $\sim 5 $ AU or further, will emit the radio waves from the system. 
The detectable parameter space will be presented in more detail in the next section. 

%%%%%%%%%%%%%%%%%%%%%%%%%%%%%%%%%%%%%%%%%%%%%%%%%%%%%%%%%%%%%%%%%%%
\subsection{Flux of RGHJ radio emission in comparison with canonical HJs}
\label{ss:brightness}
%%%%%%%%%%%%%%%%%%%%%%%%%%%%%%%%%%%%%%%%%%%%%%%%%%%%%%%%%%%%%%%%%%%

%%%%%%%%%%%%%%%%%%%%%%%%%%%%%%%%%%%
\begin{figure*}[bp]
	\plotonesc{radio_emission_ChristensenAubert_multiplecriteria.pdf}
   \caption{Radio flux in unit of Jy from a planetary companion to a red giant with mass loss rate $10^{-8} M_{\odot }/\mbox{yr}$ (top) and an AGB star with mass loss rate $10^{-5} M_{\odot }/\mbox{yr}$ (bottom). The systems are located at 100 pc away. The doubly hatched regions show the parameter space where the planetary radio emission would not be observable at all because the maximum frequency of the emission (cyclotron frequency at the planetary surface, $\nu_{\rm cyc}$) is below the plasma frequency cut-off, $\nu_{\rm plasma}^{\rm los}$. The hatched regions with vertical lines show the parameter space where the frequencies of bulk radio emission, estimated as $\sim 0.1 \nu_{\rm cyc}$, is below $\nu_{\rm plasma}^{\rm los}$, i.e., the large part of the emission would not be able to get out of the system. }
  \label{fig:radio}
\end{figure*}
%%%%%%%%%%%%%%%%%%%%%%%%%%%%%%%%%%% 

The magnetic stand-off radius (equation (\ref{eq:stand-off-radius})) may be written as follows using the stellar mass-loss rate: 
%%%
\begin{eqnarray}
\nonumber r_{\rm mag} 
&=& r_{\rm mag,\,J} \left( \frac{B}{10~\mbox{G}} \right)^{1/3} \left( \frac{a}{5.2\mbox{AU}} \right)^{1/3}  \\
&& \times \left( \frac{\dot M_\star}{\dot M_\odot} \right)^{-1/6}  \left( \frac{v}{v_{\odot }} \right)^{-1/6} \label{eq:stand-off-radius2}
\end{eqnarray}
%%%
The typical value for RGHJs is found by substituting relevant values for stellar wind parameters described in Section \ref{ss:stellarwind}:
%%%
\begin{eqnarray}
\nonumber r_{\rm mag} &\sim & 15 \, R_J \left( \frac{B}{10~\mbox{G}} \right)^{1/3}  \left( \frac{a}{5\mbox{AU}} \right)^{1/3} \\
&& \times \left( \frac{\dot M_\star}{10^{-8} M_{\odot }{\rm /yr}} \right)^{-1/6}  \left( \frac{v}{10^{-1}v_{\odot }} \right)^{-1/6}
 \label{eq:stand-off-radius_RGHJ}
\end{eqnarray}
%%%

Substituting equation (\ref{eq:stand-off-radius2}) to equation (\ref{eq:Pinp_kin}), the scaling of the radio emission is expanded as follows:
%%%
\begin{eqnarray}
P_{\rm k,\,inp} &=& nv^3 r_{\rm mag}^2 \\
&=& P_{\rm k,\,inp,\,J} \left( \frac{B}{B_{\rm J}} \right)^{2/3} \left( \frac{a}{5.2~\mbox{AU}} \right)^{-4/3}  \notag \\
&& \times \left( \frac{\dot M_\star}{\dot M_{\odot}} \right)^{2/3} \left( \frac{v}{v_{\odot}} \right)^{5/3} \label{eq:Pkinp}
\end{eqnarray}
%%%

We may compare radio emission power of canonical hot Jupiters set at 0.05 AU  and that of RGHJs at 5 AU.
Then, equation (\ref{eq:Pkinp}) can be re-normalized as follows.
%%%
\begin{eqnarray}
P_{\rm k,\,inp} 
&\approx &  140 ~P_{\rm k,\,inp,\,J} \left( \frac{B}{B_{\rm J}} \right)^{2/3} \left( \frac{a}{5~\mbox{AU}} \right)^{-4/3} \notag \\
&& \times \left( \frac{\dot M_\star}{10^{-8} M_{\odot}{\rm /yr}} \right)^{2/3} \left( \frac{v}{10^{-1} v_{\odot}} \right)^{5/3} \\
&& \;\;\;\;\;\;\;\;\;\;\;\;\;\;\;\;\;\;\;\;\; \mbox{(for RGB stars' companions)} \notag \\
&\approx & 14000 \times 10^4 ~P_{\rm k,\,inp,\,J} \left( \frac{B}{B_{\rm J}} \right)^{2/3} \left( \frac{a}{5~\mbox{AU}} \right)^{-4/3} \notag \\
&& \times \left( \frac{\dot M_\star}{10^{-5} M_{\odot}{\rm /yr}} \right)^{2/3} \left( \frac{v}{10^{-1} v_{\odot}} \right)^{5/3}  \\
&& \;\;\;\;\;\;\;\;\;\;\;\;\;\;\;\;\;\;\;\;\; \mbox{(for AGB stars' companions)} \notag \\
&\approx & 490 ~P_{\rm k,\,inp,\,J} \left( \frac{B}{B_{\rm J}} \right)^{2/3} \left( \frac{a}{0.05~\mbox{AU}} \right)^{-4/3} \notag \\
&& \times \left( \frac{\dot M_\star}{\dot M_{\odot}} \right)^{2/3} \left( \frac{v}{v_{\odot}} \right)^{5/3} \\
&& \;\;\;\;\;\;\;\;\;\;\;\;\;\;\;\;\;\;\;\;\; \mbox{(for canonical hot jupiters)} \notag 
\end{eqnarray}
%%%
Here we normalized the strength of magnetic field of canonical hot jupiters with $B_J$, due to the uncertainty of the magnetic field of tidally-locked planets. 
Depending on the modeling of magnetic field strength, there is also speculation that the tidally-locked planets may have weaker magnetic fields due to slow rotation \citep[e.g.][]{griesmeier2004}; in that case the radio emission would be weaker. 



Combining the expressions above, we find that the radio spectral flux density observed at the Earth is:
%%%
\begin{eqnarray}
F_{\nu} &=& \frac{P_{\rm radio}}{\Omega d^2 \nu_{\rm cyc}} \\
&\approx & 5.2\times 10^{-8} \mbox{Jy} \left( \frac{d}{100\mbox{~pc}} \right)^{-2}  \notag \\
&&\times \left( \frac{B}{B_{\rm J}} \right)^{-1/3}  \left( \frac{a}{5~\mbox{AU}} \right)^{-4/3} \notag \\
&& \times \left( \frac{\dot M_\star}{\dot M_{\odot}} \right)^{2/3} \left( \frac{v}{v_{\odot}} \right)^{5/3} \label{eq:F_nu} \\
&& \;\;\;\;\;\;\;\;\;\;\;\;\;\;\;\;\;\;\;\;\; \mbox{(for Jupiter-twin)} \notag \\
%%%
&\approx & 0.70 \times 10^{-5} \mbox{Jy} \left( \frac{d}{100\mbox{~pc}} \right)^{-2}  \notag \\
&&\times \left( \frac{B}{B_{\rm J}} \right)^{-1/3} \left( \frac{a}{5~\mbox{AU}} \right)^{-4/3} \notag \\ 
&& \times \left( \frac{\dot M_\star}{10^{-8} M_{\odot}/{\rm yr}} \right)^{2/3} \left( \frac{v}{10^{-1} v_{\odot}} \right)^{5/3} \label{eq:F_nu_RGHJ} \\
&& \;\;\;\;\;\;\;\;\;\;\;\;\;\;\;\;\;\;\;\;\; \mbox{(for RGB stars' companions)} \notag \\
&\approx & 0.70 \times 10^{-3} \mbox{Jy} \left( \frac{d}{100\mbox{~pc}} \right)^{-2}  \notag \\
&&\times \left( \frac{B}{B_{\rm J}} \right)^{-1/3} \left( \frac{a}{5~\mbox{AU}} \right)^{-4/3} \notag \\ 
&& \times \left( \frac{\dot M_\star}{10^{-5} M_{\odot}/{\rm yr}} \right)^{2/3} \left( \frac{v}{10^{-1} v_{\odot}} \right)^{5/3} \label{eq:F_nu_AGB} \\
&& \;\;\;\;\;\;\;\;\;\;\;\;\;\;\;\;\;\;\;\;\; \mbox{(for AGB stars' companions)} \notag \\
%%%
&\approx & 2.4 \times 10^{-5} \mbox{Jy} \left( \frac{d}{100\mbox{~pc}} \right)^{-2}  \notag \\
&&\times \left( \frac{B}{B_{\rm J}} \right)^{-1/3} \left( \frac{a}{0.05~\mbox{AU}} \right)^{-4/3} \notag \\ 
&& \times \left( \frac{\dot M_\star}{\dot M_{\odot}} \right)^{2/3} \left( \frac{v}{v_{\odot}} \right)^{5/3} \label{eq:F_nu_RGHJs} \\
&& \;\;\;\;\;\;\;\;\;\;\;\;\;\;\;\;\;\;\;\;\; \mbox{(for canonical hot jupiters)} \notag 
\end{eqnarray}
%%%
Thus, RGHJs are expected to be intrinsically as bright as the closest hot Jupiters. 
Compared with the distant Jupiter-like planets around main sequence stars, the massive stellar wind of late red giants can increase the radio emission from planetary companions by more than 2 orders of magnitude, which allow us to explore 10 times more distant system, i.e., 1000 times more volume. This compensates the small population of evolved stars at least partially. 

Equation (\ref{eq:F_nu_RGHJ}) gives the value which is by factor smaller than the prediction of equation (5) in \citet{ignace2010} in the case of $\eta = 1$. This originates from the different scaling laws for the planetary magnetic field and that their formulation didn't incorporate the effect of compressed planetary magnetosphere due to massive stellar wind. 

Using the model for planetary magnetic field strength, we may consider the  radio spectral intensity and maximum frequency of given planetary mass and age. 
Figure \ref{fig:radio} shows the contours of spectral radio flux of planetary companions at 4.5~Gyr with varying masses and orbital distances. 
The target system is located at a distance of 100~pc from the Earth; the flux is scaled by the distance as a quadratic function. 

Observable energy flux is, however, limited by the plasma cut-off frequency in the vicinity of the planets, given by equation (\ref{eq:fplasmalos_scaled}). 
The doubly hatched regions in Figure \ref{fig:radio} indicate the parameter spaces where the criteria $\nu_{\rm cyc,\,max} > \nu_{\rm plasma}^{\rm los}$ is not satisfied and thus the radio emission from the planet cannot be observed from the Earth. 
In reality, the peak of the auroral radio emission does not always occur at the maximum frequency. Instead, it usually exits at lower frequencies. Here, we also consider one more criteria, $0.1 \nu_{\rm cyc,\,max} > \nu_{\rm plasma}^{\rm los}$ as a conservative measure for the observability of the bulk of radio emission. In Figure \ref{fig:radio}, this conservative region is shown as hatched regions with vertical lines. 
It is now shown that the plasma cut-off due to massive stellar wind is a major obstacle in detecting planetary companions around evolved stars. 
For a red-giant system, the most promising targets are massive companions with $M_p \gtrsim 4M_{\rm J}$, while the smaller ones at distant orbits are also accessible.  
On the other hand, for a AGB-star system, only very massive planets at distant orbits are marginally detectable. 


%%%%%%%%%%%%%%%%%%%%%%%%%%%%%%%%%%%%%%%%%%%%%%%%%%%%%%%%%%%%%%%%%%%
\section{Observability}
\label{s:observability}
%%%%%%%%%%%%%%%%%%%%%%%%%%%%%%%%%%%%%%%%%%%%%%%%%%%%%%%%%%%%%%%%%%%

In this section, we discuss the observability of the estimated radio emission. 
An obvious potential obstacle is the intrinsic radio emission from host red-giant stars themselves; if they are bright relative to the radio flux from the planets, it is significantly more difficult identify the planetary contribution.
We will see in Section \ref{ss:RGradio} that radio flux from the planets will probably be larger in a certain parameter range. 
Then, we estimate the radio emission of Jupiter-twin around known M giants in Section \ref{ss:actualMgiants}. 
In Section \ref{ss:detectability}, we estimate the sensitivities and limitations of several current and future
radio instruments at relevant frequencies.
Finally, Section \ref{ss:timevariability} points out the polarization and the time variability of the radio emission as keys to discern the signals from the background noise. 


%%%%%%%%%%%%%%%%%%%%%%%%%%%%%%%%%%%%%%%%%%%%%%%%%%%%%%%%%%%%%%%%%%%
\subsection{Intrinsic radio emission of red giants stars}
\label{ss:RGradio}
%%%%%%%%%%%%%%%%%%%%%%%%%%%%%%%%%%%%%%%%%%%%%%%%%%%%%%%%%%%%%%%%%%%


%%%%%%%%%%%%%%%%%%%%%%%%%%%%%%%%%%%
\begin{figure*}[tbp]
   \plotonesc{cartoon_10Mj_100pc_smooth.pdf}
   \caption{A cartoon of radio emission spectra of a RGHJ with 10$M_{\rm J}$ and the host red giants with 100 $R_{\odot }$.
The spectrum of RGHJ is modeled after Jovian radio spectra, e.g. figure 8 of \citet{zarka_et_al2004} and figure 2 of \citet{griesmeier2007a}; contribution from Io is not shown here.
The spectra of the host red giant are modeled simply by extrapolating observed radio spectra above 1~GHz with power law. }
  \label{fig:cartoon}
\end{figure*}
%%%%%%%%%%%%%%%%%%%%%%%%%%%%%%%%%%% 



Over the last four decades there have been numerous radio continuum observations of nearby red giants, many with the aim of understanding the extended atmospheres and mass-loss mechanisms of K- and M-type giants~\citep[e.g.,][]{Newell1982, Knapp1995, Skinner1997, Lim1998, OGorman2013}.
Almost all observations have been carried out 
at frequencies $\geq 5~\rm GHz$, 
probing thermal Bremsstrahlung emission from the large, partially ionized envelopes surrounding the giant stars~\citep{drake1986}.
The spectral index of this emission is of order unity, with a range of reported values for various sources between 0.8 and 1.6~\citep{OGorman2013}.

Only two published studies describe attempts to detect continuum emission from single (non-binary) red giant stars 
at frequencies below 1 GHz ($L$ band); both yielded null results.
First, a program that observed a sample of nine M-type giants at 430 MHz with the Arecibo Telescope failed to detect any of the sources down to flux density 10 mJy~\citep{Fix1976}.
Later, with the Molonglo Observatory Synthesis Telescope (MOST), a sample of eight K- and M-type giants were observed at 843 MHz; the upper limits of their flux densities were placed at approximately 1 mJy~\citep{Beasley1992}.
Current facilities could achieve far deeper sensitivity on similar targets.
For example, as listed in Table~\ref{tab:sens}, the Giant Metrewave Radio Telescope (GMRT) at 150 MHz reaches a sensitivity better than 0.5~mJy in under an hour.
However, any re-attempts to detect single red giants at wavelengths near 1 meter remain to be presented in the literature.

The only red giants that have been detected at meter wavelengths are those in interacting binary systems like those of the RS CVn type.
For example, van den Oord and de Bruyn detected plasma maser emission from II Pegasi at 360 MHz and 609 MHz with the WSRT~\citep{vandenOord1994}.
Although single dwarf-type main sequence stars are also known to exhibit bright, coherent radio flares~\citep{Bastian1990}, no analog to this non-thermal emission process has been proposed to exist for evolved stars.

Altogether, we lack firm observational constraints on the brightnesses of M giants near 30-300 MHz frequencies. 

However, assuming thermal Bremsstrahlung emission continues to dominate in this wavelength regime, we can extrapolate down from the reported centimeter flux densities, using the specific spectral index. 
For example, \citet{OGorman2013} reported $\alpha$~Boo ($R_\star = 25.4 R_{\odot }$, $T=4286$ K, $d=11.26$ pc) at 1~GHz is about 70 $\mu$Jy, and $\alpha$~Tau ($R_\star = 44.2  R_{\odot }$, $T=3910$ K, $d=20.0$ pc) at 3~GHz is about 40 $\mu$Jy. 
Using these data, we extrapolate the radio flux from the host star $F_{\star}$ to the lower frequency range with a simple power law as follows:
%%%
\begin{eqnarray}
\nonumber F_{\star}(\nu ) &\sim & (0.4-1.4) \times 10^{-5} ~\mbox{Jy} \times \\
&& \left( \frac{d}{100 ~\mbox{pc}} \right)^{-2} \!\! \left( \frac{R_{\star }}{100~R_{\odot}} \right)^2 \!\! \left( \frac{\nu}{1~\mbox{GHz}} \right)^{\alpha_* } 
\end{eqnarray}
%%%
where the power index $\alpha_* $ is order of unity. 

Figure \ref{fig:cartoon} is a cartoon for a radio spectrum of a planet with $10~M_{\rm J}$ and that of the host red giant star with 100~$R_{\odot }$, both of which are placed at a distance of 100 pc. 
Note that the thermal contribution from the accretion is negligible in this figure. 
The spectral shape is modeled by simply scaling the Jovian radio spectrum (e.g. Figure 8 of \cite{zarka_et_al2004}). 
The continuum lines from the star in the cases of $\alpha_* =0.8$ and $2.0$ are shown as an extreme case.  
As Figure \ref{fig:cartoon} indicates, with a reasonable range of spectral index, the radio emission from the star is smaller by order of magnitude than the expected radio flux from RGHJs at frequencies below 300~MHz. 
For less massive planets (with weaker magnetic fields), the distinction is even clearer, because the peak flux is larger and the peak frequencies are smaller. 


Therefore, in the following, we ignore the radio emission from the red giant as a noise source. 



%%%%%%%%%%%%%%%%%%%%%%%%%%%%%%%%%%%%%%%%%%%%%%%%%%%%%%%%%%%%%%%%%%%
\subsection{Radio flux from nearby M giants}
\label{ss:actualMgiants}
%%%%%%%%%%%%%%%%%%%%%%%%%%%%%%%%%%%%%%%%%%%%%%%%%%%%%%%%%%%%%%%%%%%


In order to examine the potential to detect radio emission of RGHJs with the current and near-future instruments, we estimate the possible radio emission from known red giants assuming that they have planetary companions. 
For this purpose, we obtained the list of M-type red giants from Table 4 of \citet{dumm1998} with the data of mass, radius, and effective temperature.
For each red giant, the mass-loss rate is estimated by the improved Reimers' equation \citep{reimers1975} given by \citet{schroder2005,schroder2007}:
%%%
\begin{equation}
\dot M_\star [M_\odot{\rm /yr}] \sim 8 \times 10^{-14} \frac{\tilde L \tilde R}{\tilde M} \left( \frac{T_{\rm eff}}{4000} \right)^{3.5} \left( 1 + \frac{g_{\odot }}{4300 g_{\star}} \right) \label{eq:mass-loss}
\end{equation}
%%%
where $\tilde L = L_{\star }/L_{\odot }(=4\pi R_{\star }^2 T_{\star }^4)$, $\tilde R = R_{\star }/R_{\odot }$, $\tilde M = M_{\star }/M_{\odot }$, and $g$ is the surface gravity. 
The velocity of the stellar wind is assumed to be the escape velocity at $\sim 4 R_{\star }$:
%%%
\begin{equation}
v \sim \sqrt{\frac{2GM_\star}{4R_{\star }}}
\end{equation}
%%%
The age of the system is assumed to be roughly the same as the lifetime of the main-sequence stage, which depends on the stellar mass. We simply assume:
%%%
\begin{equation}
t \sim 10\,{\rm Gyr}\, \left( \frac{M_{\star}}{M_{\odot }} \right)^{-2.5}
\end{equation}
%%%
Combining this relation with equations (\ref{eq:fcyc}) and (\ref{eq:scalingB}) results in
%%%
\begin{equation}
\nu_{\rm cyc,\,max} \sim 20 \,{\rm MHz}\, \left( \frac{M_{\star}}{M_{\odot }} \right)^{1.075} \left( \frac{M_p}{M_{p,\,{\rm J}}} \right)^{1.04} 
\end{equation}
%%%
In addition, the distance to the system is obtained based on the parallax data from Hipparcos datasets\footnote{http://www.rssd.esa.int/index.php?project=HIPPARCOS}.

Using these data, we calculate the spectral flux of radio emission by specifying planetary mass and orbital distance of a hypothetical planetary companion. 
We consider [A] a Jupiter-twin whose maximum cyclotron frequency is $\nu _{\rm cyc,\,max} \sim 20\,(M_{\star }/M_{\odot})$~MHz, and [B] a larger Jovian planet with $M_p\sim 10M_p$ whose cyclotron frequency is $\nu _{\rm cyc,\,max} \sim 200\,(M_{\star }/M_{\odot})$~MHz. 

Figure \ref{fig:observability} displays the estimated radio flux from Jupiters [A] and [B] around M-type red giants within 300~pc. 
Jupiters are placed at 1 AU and at 5 AU for reference. 
The difference of the symbols indicates whether the radio emission can escape from the system (circle), i.e., $\nu _{\rm cyc,\,max} > \nu_{\rm plasma}^{\rm los}$,  or not (cross). 
Assuming Jupiter-mass planets which typically have the maximum cyclotron frequency at $10-100$~MHz depending on the age, a reasonable number of systems have potential to emit radio wave at the level above $\sim 10~\mu {\rm Jy}$  out to $\sim 100-300$ pc, while the those with the highest mass-loss rate would suffer from the plasma cut-off of the stellar wind. 
For more massive planets with the maximum cyclotron frequency at around $100-1000$~MHz, the radio flux is lowered due to its dependence on $B^{-1/3}$ (equation (\ref{eq:F_nu})), but more likely to get to the Earth. $10~M_{\rm J}$ planets at $\sim 1$ AU will be detectable out to $\sim 200$~pc.


%%%%%%%%%%%%%%%%%%%%%%%%%%%%%%%%%%%
\begin{figure}[tbhp]
   \plotoneh{radio_M-RG_1Mp.pdf}
   \plotoneh{radio_M-RG_10Mp.pdf}
   \caption{Radio emission from existing M-type red giants with a hypothetical planetary companion of 1$M_{\rm J}$ (upper panel) and 10$M_{\rm J}$ (lower panel) at 1~AU (blue) and 5~AU (black).
Cross symbols indicate that the radio emission is not observable because cyclotron frequency is less than plasma density around the planet. }
  \label{fig:observability}
\end{figure}
%%%%%%%%%%%%%%%%%%%%%%%%%%%%%%%%%%% 


%%%%%%%%%%%%%%%%%%%%%%%%%%%%%%%%%%%%%%%%%%%%%%%%%%%%%%%%%%%%%%%%%%%
\subsection{Detectability with current and near-future low-frequency radio instrumentation}
\label{ss:detectability}
%%%%%%%%%%%%%%%%%%%%%%%%%%%%%%%%%%%%%%%%%%%%%%%%%%%%%%%%%%%%%%%%%%%

%%%%%%%%%%%%%%%%%%%%%%%%%%%%%%%%%%%%%%%%%%%%%%%%%%%%%%%%%%%%%%%%
\begin{deluxetable*}{cccccc}
\tabletypesize{\footnotesize}
\tablecolumns{6}
\tablecaption{Detectability of Hot Jupiters with Current 
and Future Radio Telescopes.\label{tab:sens}}
\tablehead{
Instrument & Band  & \sigRMS\tablenotemark{a} & $\tau$\tablenotemark{b}  & Resolution & $\sigma_c$\tablenotemark{c}\\
           & (MHz) & ($\mu$Jy bm$^{-1}$)         &  (hr)                    & (\arcsec)  & ($\mu$Jy bm$^{-1}$)} 
\startdata
LOFAR-LBA  &  30-80  & 1100  & 230000             & 10   & 57 \\ 
LOFAR-HBA  & 110-200 & 180   &   6500             & 4.1  & 1.1 \\  
HERA       & 50-250  &  76   & 1200               & 410  & 36000\\ 
GMRT       & 130-190 & 162   &   5400             & 16.5 & 57 \\ 
%LWA       &  30-80  & 50000 &  702127141         & 10.3 & 42.2 \\
%MWA       &  30-80  & 24000 &  176223211         & 1030 & 4.2$\times10^5$ \\
SKA-Low (part) & 50-200 & 5.9 &   7.2             & 7.6  & 9.7 \\
SKA-Low (full) & 50-350 & 1.8 &  0.7              & 4.8  & 1.5 \\
%VLA 4-Band& 30-80   & 3300  & 2312058            & 28.6 & 330 \\
VLA-LOBO   &  50-350 &  77   &  1200              & 8.6  & 10.4
\enddata
%\tablecomments{Time estimates to reach 2~$\mu$Jy RMS.}
\tablenotetext{a}{RMS noise in a 1-hour observation, computed for imaging of the full band reported 
in this table (Equation~\ref{eq:rms_sens}).}
\tablenotetext{b}{The integration time in hours required to reach an RMS of 2~$\mu$Jy (i.e.\ 5-$\sigma$ detection
of a typical source at 100~pc with 10~$\mu$Jy bm$^{-1}$ average flux).}
\tablenotetext{c}{Source confusion RMS contribution.}
\end{deluxetable*}
%
% HERA
% http://reionization.org/wp-content/uploads/2015/05/HERA0_imaging.pdf
%
% LOFAR
% https://www.astron.nl/~heald/test/sens.php; 
% 24 core stations, 16 remote, 8 international, 488 subbands
%
% GMRT
% http://gmrt.ncra.tifr.res.in/~astrosupp/obs_setup/sensitivity.html
%
%%%%%%%%%%%%%%%%%%%%%%%%%%%%%%%%%%%%%%%%%%%%%%%%%%%%%%%%%%%%%%%%

We consider in this section several current and near-future radio observatories that operate at frequencies $\nu \lesssim 350~\rm MHz$ and can reach continuum sensitivities less than 1~mJy in a few hours.
The results are summarized in Table~\ref{tab:sens}. 

The primary limitation to detectability is a telescope's sensitivity, characterized by the root mean square (RMS) noise fluctuations in the sky and receiver.
The RMS noise \sigRMS\ in an interferometric observation with integration time $\tau$, bandwidth $\Delta \nu$, effective area $A_{\rm eff}$ ($\approx0.7 \pi (D/2)^2$ for interferometers comprising antennas with dish size $D$), and $N$ antennas is \citep[see e.g.][]{condon+ransom2016}
%%%
\begin{equation}
\sigRMS = \frac{2 k \Tsys}{A_{\rm eff} \, \sqrt{\Delta \nu \, \tau N (N-1)}} 
\left(\frac{\rm 10^{26} Jy}{\rm W m^{-2} Hz^{-1}}\right).
\label{eq:rms_sens}
\end{equation}
%%%
Here, \Tsys\ is the blackbody temperature equivalent of the total system noise, which is the sum of the instrument or receiver noise temperature \Trx\ and the noise contribution from the sky \Tsky.
We estimate \Tsky\ using the low frequency sky noise temperature fit from \cite{Rogers+Bowman2008}:
%%%
\begin{equation}
\Tsky = T_{150} \left( \frac{\nu}{150~\rm MHz}\right)^{-\beta} + \Tcmb
\label{eq:Tsky}
\end{equation}
%%%
where $T_{150} = 283.2~\rm K$ is the \Tsky\ at 150~MHz, $\beta=2.47$, and $\Tcmb = 2.73~\rm K$ is the contribution from the cosmic microwave background (CMB).
While this relation was fit to data taken in the 100-200~MHz range, the authors found it to be consistent with published measurements from 10-408~MHz. 
%The \Trx\ is the receiver noise temperature, which depends on the instruments and is described below. 

An additional factor that affects detectability is source confusion, the imaging noise associated with unresolved interloping radio sources in an observation.
Source confusion is characterized by the standard deviation $\sigma_c$ in the surface brightness of an image due to one or more unresolved sources in the beam solid angle (for a review, see \cite{Condon1974,Condon2012}).
To estimate $\sigma_c$, reported in Table~\ref{tab:sens}, we use the relations provided by \cite{condon+ransom2016}, reproduced below:
%%%
\begin{eqnarray}
\sigma_c & \approx & 0.2 \left( \frac{\nu}{1 \rm~GHz} \right)^{-0.7} \left( \frac{\theta}{1\arcmin} \right)^2\!\!\!\!,~(\theta > 10\arcsec) \\
\sigma_c & \approx & 2.2 \left( \frac{\nu}{1 \rm~GHz} \right)^{-0.7} \left( \frac{\theta}{1\arcmin} \right)^{10/3}\!\!\!\!\!\!\!\!\!\!,~(\theta < 10\arcsec).
\end{eqnarray}
%%%
As Table~\ref{tab:sens} illustrates, source confusion is likely to exceed the 2 $\mu$Jy detection limit of a typical RGHJ source for most
telescopes under consideration.  However, source confusion can be mitigated through differencing observations that cancel out background sources.
Potential axes for such cancellation are polarization and time.
The source confusion limit is dominated by active galactic nuclei, which have low polarization fractions $\lesssim2.5\%$ \citep{Stil2014} and slowly vary with time.
Therefore both the circularly polarized nature of the cyclotron emission from planets and the time variability (discussed in \S\ref{ss:timevariability}) partially mitigate the source confusion limit, allowing planetary radio emission, like other time-variable sources, to be detected below the imaging confusion limit.
We therefore expect confusion not to be a strong limitation to the detectability of radio emission from RGHJs.

% LOFAR
The LOw-Frequency ARray (LOFAR) operates as two separate arrays defined by their high and low bands, differing configurations, and antenna designs, which are called respectively the Low Band Antennas (LBA) and High Band Antennas (HBA).
LOFAR-LBA operates in the 10-90~MHz range, and LOFAR-HBA operates in the 100-240~MHz range \citep{vanHaarlem2013}.
We consider it as two separate instruments here.
LOFAR-LBA's most sensitive band is centered at 60~MHz, and LOFAR-HBA's is at 150~MHz.
Given the strong frequency dependence of the effective aperture of a dipole antenna, and that the reciever noise temperature for LOFAR $\Trx \sim \Tsky$, we use the LOFAR Image Noise calculator\footnote{Heald; \url{http://www.astron.nl/~heald/test/sens.php}}, which is based on {\it SKA Memo 113}\footnote{Nijboer, Pandey-Pommier, \& de Bruyn; \url{http://www.skatelescope.org/uploaded/59513\_113\_Memo\_Nijboer.pdf}}, to obtain the sensitivity and imaging parameters in Table~\ref{tab:sens}.

% HERA
The Hydrogen Epoch of Reionization Array (HERA\footnote{\url{http://reionization.org/}}) is a telescope array under construction in South Africa that will ultimately consist of 352 14-m parabolic dishes, operating from 50 to 250~MHz \citep{pober_et_al2014}.
Unlike the other telescope observatories considered here, HERA is a dedicated experiment for making power spectral measurements of cosmic reionization over wide fields of view.
Similar to its progenitor, the Precision Array to Probe the Epoch of Reionization (PAPER; \citealt{parsons_et_al2010}), HERA's dishes are deployed in an extremely compact configuration \citep{parsons_et_al2012} and statically point toward zenith.
While these features make HERA less ideal for targeted observations, HERA's substantial collecting area, long observing campaigns, and wide field of view make it a powerful survey instrument.

% GMRT
The Giant Metre-wave Radio Telescope (GMRT) is an array comprising thirty 25-meter antennas, operating at frequencies 130-2000~MHz.
The lowest band (130-190~MHz) exhibits receiver noise $T_{\rm rx}$ comparable to the sky noise.
To be conservative, we use the GMRT online calculator.\footnote{\url{http://gmrt.ncra.tifr.res.in/~astrosupp/obs\_setup/sensitivity.html}}


% SKA
The low-frequency component of the Square Kilometer Array (SKA), recently rebaselined to have 0.4~km$^2$ collecting area in 130,000 elements with baselines extending to 65~km, will operate from 50-350~MHz.\footnote{\url{https://www.skatelescope.org/news/worlds-largest-radio-telescope-near-construction/}}
We estimate the noise in the lower half of the band ($\nu<200~\rm MHz$), appropriate for lower mass ($\lesssim 10 M_{\rm J}$) separately from the full band, which could be used to detect $\sim 30 M_{\rm J}$ Jupiters.
We again adopt the \cite{Rogers+Bowman2008} relation for \Tsys\ (Equation~\ref{eq:Tsky}), and assume $\Trx \approx 60~\rm K$.
We caution that these assumptions are optimistic, and the time estimates could in reality be a factor of a few worse.

% VLA
For the low frequency receivers (28-80~MHz; 4-Band) in the Very Large Array (VLA) have $\Trx \sim 260~\rm K$, similar to the receiver noise temperatures for the Long Wavelength Array (LWA) \citep{Hicks2012}.  As noted in \citep{Hicks2012}, this noise level is subdominant to \Tsky\ by at least -6~dB.
Future upgrades to the VLA such as the LOw Band Observatory \citep[LOBO][]{Kassim2015IAU} are being considered, and could potentially cover the full 30-470~MHz band.
We estimate the sensitivity of LOBO assuming the receiver noise extrapolates linearly from $\Trx \approx 60~\rm K$ (internal VLA memo) in P-Band (230-470~MHz) to $\Trx \approx 260~\rm K$ in 4-band, and consider just the 50-350~MHz portion of the full 30-470~MHz band.


The noise estimates shown in Table \ref{tab:sens}, together with Figure \ref{fig:observability}, indicate that, at the frequencies relevant for the detection of radio emission from Jupiter-size planets, current instrument sensitivities are generally too low.
%On the other hand, for more massive exoplanets ($M_p\sim 10 M_{\rm J}$) where the bulk of the radio emission would exist above 50~MHz, there is hope to detect radio emission from nearby systems with LOFAR-HBA or LOBO.
%LOFAR-HBA has reasonable sensitivity to $M_p =10~M_{\rm J}$ planets at 5 AU around the few closest systems as well as those around further stars but orbiting closer to the stars. 
On the other hand, for more massive exoplanets ($M_p\sim 10 M_{\rm J}$) where the bulk of the radio emission would exist above 50~MHz, low frequency component of SKA will be able to detect the signal within one day from planets with $M_p \sim 10~M_{\rm J}$ at $\sim$1~AU and at $\sim$5~AU within $\sim $200~pc and $\sim $100~pc from Earth, respectively.  
VLA-LOBO could also work for a few nearby systems. 




%%%%%%%%%%%%%%%%%%%%%%%%%%%%%%%%%%%%%%%%%%%%%%%%%%%%%%%%%%%%%%%%%%%
\subsection{Discriminating the signal from the background}
\label{ss:timevariability}
%%%%%%%%%%%%%%%%%%%%%%%%%%%%%%%%%%%%%%%%%%%%%%%%%%%%%%%%%%%%%%%%%%%


Since the confusion limit is on the same order as the signal (Table \ref{tab:sens}), it is crucial to consider the ways to discriminate the signal from these sources. 
In this section, we discuss two key features to identify the auroral radio emission from the planets. 

One of the key features that makes planetary radio emission distinct is the nearly  complete circular polarization \citep[e.g.][and reference therein]{dessler1983}. 
On the other hand, active galactic nuclei, which are the major confusion sources, are known to have low polarization fractions $\lesssim2.5\%$ \citep{Stil2014}. 

Another key is the significant time variability of the planetary radio emission. 
The factors that influence time variability are listed below. 

\paragraph{Modulation due to planetary spin rotation (a few hours $\sim $)}
The probability of the occurrence of the Jovian radio burst is correlated with the spin rotation phase \citep[e.g.][]{dessler1983} due to the misalignment between the magnetic axis and the spin axis. If the time resolution of the observations is high enough, we might detect the periodicity due to planetary spin rotation. 
%
\paragraph{Modulation due to the presence of satellites (a few days $\sim $)}
The occurrence probability of Jovian radio burst is affected by the location of Io as well \citep{dessler1983}, because Io disturbs the surrounding electromagnetic field. 
Likewise, if the exoplanets have moons that emit plasmas, then the radio flux is  modulated by the orbital motion of the moon \citep[see e.g.,][]{noyola2014}. 

\paragraph{Variability of the stellar wind (a few months $\sim $?)}
If the stellar wind passing the planet is variable, any modulation of the plasma density would also lead to time variabililty of planetary radio emission.  Such variability might be expected to occur on timescales of several months to half-a-year as many M giants have semi-regular periodicities on the order of a few hundred days \citep{Kiss:1999aa}.

%
\paragraph{Modulation due to orbital motion of the planet (a few years $\sim $)}
If the planetary orbit is close to the edge-on configuration, the radio emission disappears concurrently with the planetary orbital motion, due to the increased plasma cut-off frequency along the path and/or secondary eclipse \citep{lecavelier_et_al2013}. 
The former is because when the planet is behind the star the path of the radio emission toward the Earth comes through the vicinity of the star where the plasma density is high.
The probability of the latter one, secondary eclipse, is $\sim R_\star/a \sim 9$\% (in the case that $R_{\star}=100R_{\odot }$ and $a=5~\mathrm{AU}$). 
%



%%%%%%%%%%%%%%%%%%%%%%%%%%%%%%%%%%%%%%%%%%%%%%%%%%%%%%%%%%%%%%%%%%%
\section{Discussion}
\label{s:discussion}
%%%%%%%%%%%%%%%%%%%%%%%%%%%%%%%%%%%%%%%%%%%%%%%%%%%%%%%%%%%%%%%%%%%


%%%%%%%%%%%%%%%%%%%%%%%%%%%%%%%%%%%%%%%%%%%%%%%%%%%%%%%%%%%%%%%%%%%
\subsection{Implication for discovery and characterization of exoplanets}
\label{ss:implication}
%%%%%%%%%%%%%%%%%%%%%%%%%%%%%%%%%%%%%%%%%%%%%%%%%%%%%%%%%%%%%%%%%%%


So far, more than 60 exoplanets are found around evolved stars \citep[e.g.][and references therein]{trifonov2015}. 
Therefore, one possible strategy is to target known planetary systems.  
One complication of the radial velocity observations of evolved stars is the high degree of jitter due to chromospheric activities and stellar pulsations, which typically produces RV errors of $\sim 30$ [m/s]. 
Therefore, there are some parameter spaces where current radial velocity observations have not been able to survey. For such a parameter space, radio emission would work as a  probe to discover the possible sub-stellar companions. 

Once radio emission is detected, it provides some useful information about the planet itself and the stellar wind through the time dependence and frequency dependence. 
As mentioned in \S\ref{ss:timevariability}, auroral radio flux significantly varies in time in association with several physical processes. In turn, we could utilize the time variability to characterize the system. For example, if the time resolution of the observation is sufficiently high (with full SKA, for instance), the time variation associated with planetary spin rotation could be detectable. 
Long term monitoring could reveal the time variability corresponding to orbital period if the system is near edge-on. Modulation on timescale of a few days may imply the presence of satellites. 

If the spectra are obtained, the upper cut-off frequency of the radio spectra will indicate the magnetic field strengths of the planetary surface. Such information would be most useful if we could obtain constraints on planetary mass via e.g. radial velocity observations, because once both magnetic field strength and the planetary mass is constrained the proposed scaling law of planetary dynamo may be tested. On the other hand, the lower cut-off frequency indicates the plasma frequency in the vicinity of the planets, which might be used as a probe of the stellar wind property at the planetary orbit. 


%%%%%%%%%%%%%%%%%%%%%%%%%%%%%%%%%%%%%%%%%%%%%%%%%%%%%%%%%%%%%%%%%%%
\subsection{Back-reaction of a plasma flowing into a magnetic field}
\label{ss:offset}
%%%%%%%%%%%%%%%%%%%%%%%%%%%%%%%%%%%%%%%%%%%%%%%%%%%%%%%%%%%%%%%%%%%

One might expect that plasma flowing into a region permeated by a magnetic field would, by spiraling around the field lines, generate an opposing magnetic field that partially cancels the intrinsic planetary magnetic field. 
To estimate the strength of this effect, consider a uniform flow of particles of charge $e$, mass $m$, and velocity $v$ into a region of magnetic field $B$, spiraling along the initial magnetic field. 
A single charged particle moving perpendicular to the magnetic field moves in a circular orbit of radius 
%%%
\begin{equation}
r=\frac{mv_\bot }{eB}.
\end{equation}
%%%
which creates a magnetic dipole
%%%
\begin{equation}
\mu = \frac{e}{2\pi r/v_\bot} \pi r^2 = \frac{1}{2} e r v_\bot = \frac{mv_\bot^2}{2B}.
\end{equation}
%%%
The volume integral of the canceling magnetic field $B_c$ generated by a single dipole $\mu$ is
%%%
\begin{equation}
\int d^3{\boldsymbol r} B_c = \frac{8\pi}{3} \mu.
\end{equation}
%%%

Hence, for a number density $n$ of these charged particles, the fraction of canceling field to the background magnetic field is
%%%
\begin{equation}
\frac{B_c}{B} = \frac{n(mv_\bot^2/2)}{3 B^2/8\pi} < \frac{\rho_K}{3\rho_B},
\end{equation}
%%%
where $\rho_K$ and $\rho_B$ are the kinetic energy density and the magnetic energy density, respectively. 

Just inside the stand-off point, 
%%%
\begin{equation}
m_p n[r_{\rm mag}] v^2 < \frac{B[r_{\rm mag}]^2}{2\pi }
\end{equation}
%%%
and thus $\rho_K/\rho _B < 1$. 
When the particle spirals into the planet, 
the magnetic pressures increases as 
%%%
\begin{equation}
\frac{B^2}{8\pi } \propto \frac{1}{r^6}
\end{equation}
%%%
On the other hand, the kinetic energy of the moment increases
%%%
\begin{equation}
\frac{mv^2}{2} \propto \frac{1}{r}
%\rho _K \propto \frac{1}{r}
\end{equation}
%%%
Therefore, unless the density increases more drastically than $1/r^5$, the ratio $B_c/B$ is significantly less than unity. 
 


%%%%%%%%%%%%%%%%%%%%%%%%%%%%%%%%%%%%%%%%%%%%%%%%%%%%%%%%%%%%%%%%%%%%%%%%
\section{Conclusions}
\label{s:conc}
%%%%%%%%%%%%%%%%%%%%%%%%%%%%%%%%%%%%%%%%%%%%%%%%%%%%%%%%%%%%%%%%%%%%%%%%

In this paper, we estimate the radio brightness of distant ``hot Jupiters'' around evolved stars (RGHJ) on the assumption of the simple empirical correlation between planetary radio emission and the input kinetic energy from the stellar wind. 
Unlike the previous paper, we consider that UV/X-ray from accretion onto the planetary companion ionizes the stellar wind in the vicinity of planets, which would otherwise be fairly neutral. 
Based on such a picture, the massive stellar wind of a red giant or a AGB star would interact with the planetary magnetic field and input kinetic energy into the magnetosphere of a RGHJ. 
We found that the intrinsic brightness of radio emission from RGHJs are supposedly comparable to canonical hot Jupiters in close-in orbits around main-sequence stars, and $>100$ times brighter than distant Jupiter-twins around main-sequence stars. This implies that they can be searched for 10 times further away, or in 1000 times larger volume. This can compensate the rareness of the evolved stars at least partly. Thus, RGHJs will serve as  reasonable targets for future search for exoplanetary radio emission. 

A major obstacle to observe this radio emission is the plasma cut-off frequency of the (ionized) stellar wind. Due to the great density of the stellar wind, the cut-off frequency is as high as $\sim 12$ MHz for typical red giants and $\sim$ 400 MHz for typical AGB stars, making planetary-mass companions to AGB stars unpromising for the search of planetary auroral radio emission. 
The most promising targets are massive planetary companions ($M_p \gtrsim 4 M_{\rm J}$ to red giant stars, with magnetic field stronger than Jupiter (i.e., higher cyclotron frequency). 

The radio flux from a system with an exoplanet at 5 AU at a distance of 100 pc would be typically on the order of $\sim 10 \mu $Jy.
Such signals would be detectable with SKA within a day. 
For a few nearby systems, VLA-LOBO would also work with a reasonable integration time.
In both cases, it is critical to consider polarization and/or time variability of planetary radio flux, in order to separate it from confused sources in the same resolution element. 

Radio emission can be follow-up observations of known RGHJs or may also be used as a probe to discover them, given the lowered sensitivity of  radial velocity methods in detecting planetary companions. 
Once the planetary radio emission is detected, it will provide a unique approach to the planetary spin, magnetic field, presence of satellites, and potentially stellar wind properties. 


\vspace{0.5in}

\acknowledgements

{\sc Acknowledgments}

YF is supported from the Grant-in-Aid No. 25887024 by the Japan Society for the Promotion of Science.
DSS gratefully acknowledges support from a fellowship from the AMIAS group. JN acknowledges support from NASA grant HST AR-12146.04-A and NSF grant AST-1102738.
We thank David Hogg for encouraging us to pursue calculations of exoplanetary radio emission.
We thank Greg Novak for helpful conversations.
NM acknowledges support from [???]. 
AP is grateful for support from NSF grants 1352519 and 1440343.
YF greatly acknowledges insightful and helpful discussions with Tomoki Kimura and Hiroki Harakawa. 
The portion of this research for which TM is responsible was performed under a National Research 
Council Research Associateship Award at the Naval Research Laboratory (NRL).
Basic research in radio astronomy at NRL is supported by 6.1 Base funding.


\bibliography{biblio.bib}


\newpage

\appendix



\section{Ionization Cascade}
\label{sec:AppendixA}

For an electron-hydrogen ionizing collision, we estimate the differential cross section $\sigma$ as a function of recoil energy $d\sigma / dE_{\rm recoil}$.
We approximate the bound electron as a free one at rest and use the differential cross section for scattering of two electrons (Bhabha scattering) in the non-relativistic limit:
\begin{equation}
\label{eq:dsig/dOmeg} \frac{d\sigma}{d\Omega_{\rm CM}} \sim \frac{\alpha^2}{m_e^2 v_{\rm rel}^2 \sin^4 \theta} \, ,
\end{equation}
where $\alpha$ is the fine structure constant, $v_{\rm rel}$ is the relative velocity, $\theta$ is the scattering angle in the center of mass (CM) frame, and $m_e$ is the electron mass.
The recoil energy (that is, the energy transferred to the electron at rest) is related to the scattering angle via
\begin{equation}
  \label{eq:Erec} = \frac{1}{4} m_e v_{\rm rel}^2 \left( 1 - \cos \theta \right) \, ,
\end{equation}
and hence
\begin{equation}
  \label{eq:dErec/dOmeg} \frac{dE_{\rm recoil}}{d\Omega_{\rm CM}} = -\frac{1}{4} m_e v_{\rm rel}^2 \, .
\end{equation}
We therefore have
\begin{equation}
  \label{eq:dsig/dErec} \frac{d\sigma}{dE_{\rm recoil}} \sim \frac{\alpha^2}{m_e v_{\rm rel}^2 E_{\rm recoil}^2} \, ,
\end{equation}
which is largest for small recoil energies.
Of course, $E_{\rm recoil}$ must be larger than $E_{\rm Rydberg}$ for this approximation to be meaningful. 

An objection to the above estimate is that the singularity in the $E_{\rm recoil} \rightarrow 0$ limit arises from the long range nature of Coulomb interaction which of course is absent in the real problem.
Far away, the ionizing electron sees the dipole moment of the neutral atom.
However, for energy transfer of order a Rydberg $\alpha/2a_0$ we can estimate the transfer momentum and see that it is larger than the inverse of the Bohr radius $a_0$.
This implies that in this range the above estimate may actually be reliable. The transfer momentum is
\begin{equation}
q^2 =\frac{1}{4} m^2 v_{\rm rel}^2 (\sin^2\theta +(1-\cos \theta)^2)= m^2v_{\rm rel}^2 \sin^2 \frac{\theta}{2}.
\end{equation}
Comparison with (\ref{eq:Erec}) shows
\begin{equation}
E_{\rm recoil} = \frac{q^2}{2m}.
\end{equation}
Now set $E_{\rm recoil} \sim \alpha/a_0$ to get
\begin{equation}
q \sim \sqrt{\frac{m\alpha}{a_0}}\sim \frac{1}{a_0}
\end{equation}
where we used $a_0 = 1/\alpha m$.


\section{Comparison with previous work}
\label{sec:AppendixB}

There is a rather wide range in estimates of the radio emission from Jovian exoplanets and it is worth noting how the different models yield different answers. 

A common part is the scaling relationship between the planetary magnetic field strength and the input kinetic energy.
From equations \ref{eq:Pinp_kin} and \ref{eq:n}, the input kinetic energy scales as
%%%
\begin{equation}
P_{\rm inp,k}=nv^3 r_{\rm mag}^2 \propto \dot M_{\star} v^2 a^{-2} r_{\rm mag}^2.
\end{equation}
%%%
Combining the above equation with the scaling between the radius of the magnetic stand-off point, $r_{\rm mag}$, and the planetary magnetic field $B$ (equation (\ref{eq:stand-off-radius})), i.e.\
%%%
\begin{equation}
r_{\rm mag} \propto B^{1/3} a^{1/3} \dot M_{\star }^{-1/6} v^{-1/6}, 
\end{equation}
%%%
and assuming a constant stellar wind, we obtain:
%%%
\begin{equation}
P_{\rm inp, k} \propto B^{2/3} a^{-4/3} \dot M_{\star }^{2/3} v^{5/3}.
\end{equation}
%%%

Aside from this common part, the differences originate mainly from:
\renewcommand{\theenumi}{\roman{enumi}}
%%%
\begin{enumerate}
\item Assumptions for the scaling law of radio power in terms of input energy. 
\item Assumptions for the scaling law of planetary magnetic field strength. 
\item Different normalization (assumptions for Jovian radio emission). 
\item Assumptions for band width (either $\nu_{\rm cyc}$ or $0.5\nu_{\rm cyc}$). 
\item Modeling of stellar wind
\end{enumerate}
%%%

As for i., ii. and iii., the scaling laws are summarized in Table \ref{tab:comp}. 

%%%%%%%%%%%%%%%%%%%%%%%%%%%%%%%%
\begin{table*}[htp]
\caption{Comparison for assumed scaling relationships.}
\begin{center}
\begin{tabular}{c|l|llcc} \hline \hline
%
& Paper & Radio power & Magnetic moment  & norm [W] & band width \\\hline
%
\#1 & \citet{farrell1999} & $P_{\rm radio} \propto P_{\rm sw}^{0.88} $ & $\mathcal{M} \propto \omega M_p^{5/3}$ & $4 \times 10^9$ & $ 0.5 \nu_{\rm cyc}$ \\ \hline
%
\#2 & \citet{farrell1999} & $P_{\rm radio} \propto P_{\rm sw}^{1.2} $ & $\mathcal{M} \propto \omega M_p^{5/3}$ & $4 \times 10^{11}$ & $ 0.5 \nu_{\rm cyc}$ \\
& \citet{lazio2004} & & $\mathcal{M} \propto \omega M_p^{2}$?? & & \\
& \citet{ignace2010} & & & & \\ \hline
\#3 & \citet{stevens2005} & $P_{\rm radio} \propto P_{\rm sw} $ & $\mathcal{M} \propto M_p$ & - & - \\ \hline
\#4 & \citet{griesmeier2005,griesmeier2007a,griesmeier2007b} & $P_{\rm radio} \propto P_{\rm sw}$ & $\mathcal{M} \propto \rho _c^{1/2} \omega r_c^4 $ & $2.1 \times 10^{11}$ & $ 0.5 \nu_{\rm cyc}$ \\
\#5 & & & $\mathcal{M} \propto \rho _c^{1/2} \omega^{1/2} r_c^3 \sigma ^{-1/2} $ & & \\
\#6 & & & $\mathcal{M} \propto \rho _c^{1/2} \omega^{3/4} r_c^{7/2} \sigma ^{-1/4} $ & \\
\#7 & & & $\mathcal{M} \propto \rho _c^{1/2} \omega r_c^{7/2}$ & & \\ \hline
%\#8 & \citet{griesmeier2007a,griesmeier2007b} & $P_{\rm radio} \propto P_{\rm sw}$ & same as above & $2.1 \times 10^{11}$ & $ \nu_{\rm cyc}$ \\ \hline
\#8 & \citet{reiners2010} & $P_{\rm radio} \propto P_{\rm sw}$ & $\mathcal{M} \propto M_p^{1/6} L^{1/3} R_p^{25/6}$ \\ 
 & this work & & \\ \hline
\end{tabular}
\end{center}
\label{tab:comp}
\tablecomments{}
\end{table*}
%%%%%%%%%%%%%%%%%%%%%%%%%%%%%%%%

If we use the assumptions that $R_p \sim R_J\;\;(const.)$, $\rho _c\sim R_p\sim const.$, and $\sigma = const.$, then the assumptions are reduced to:
%%%%%%%%%
\begin{eqnarray}
\mbox{[\#1]} \; P_{\rm rad} &\sim & 4 \times 10^9 \,{\rm W}\,\left( \frac{\omega}{\omega _{\rm J}} \right)^{0.58} \left( \frac{M_p}{M_{\rm J}} \right)^{0.98} \left( \frac{a}{a_{\rm J}} \right)^{-1.17} \left( \frac{\dot M_\star}{\dot M_{\odot}} \right)^{0.586} \left( \frac{v}{v_{\odot}} \right)^{1.46} \\
%
\mbox{[\#2]} \; P_{\rm rad} &\sim & 4 \times 10^{11} \,{\rm W}\, \left( \frac{\omega}{\omega _{\rm J}} \right)^{0.79} \left( \frac{M_p}{M_{\rm J}} \right)^{1.33} \left( \frac{a}{a_{\rm J}} \right)^{-1.60} \left( \frac{\dot M_\star}{\dot M_{\odot}} \right)^{0.8} \left( \frac{v}{v_{\odot}} \right)^{2.0} \\
%
\mbox{[\#3]} \; P_{\rm rad} &\sim & ? \times 10^{11} \,{\rm W}\,\left( \frac{M_p}{M_{\rm J}} \right)^{2/3} \left( \frac{a}{a_{\rm J}} \right)^{-4/3} \left( \frac{\dot M_\star}{\dot M_{\odot}} \right)^{2/3} \left( \frac{v}{v_{\odot}} \right)^{5/3} \\
%
\mbox{[\#4]} \; P_{\rm rad} &\sim & 2.1 \times 10^{11} \,{\rm W}\,\left( \frac{\omega}{\omega _{\rm J}} \right)^{2/3} \left( \frac{M_p}{M_{\rm J}} \right)^{1/3} \left( \frac{a}{a_{\rm J}} \right)^{-4/3} \left( \frac{\dot M_\star}{\dot M_{\odot}} \right)^{2/3} \left( \frac{v}{v_{\odot}} \right)^{5/3} \, \\
%
\mbox{[\#5]} \; P_{\rm rad} &\sim & 2.1 \times 10^{11} \,{\rm W}\,\left( \frac{\omega}{\omega _{\rm J}} \right)^{1/3} \left( \frac{M_p}{M_{\rm J}} \right)^{1/3} \left( \frac{a}{a_{\rm J}} \right)^{-4/3} \left( \frac{\dot M_\star}{\dot M_{\odot}} \right)^{2/3} \left( \frac{v}{v_{\odot}} \right)^{5/3} \, \\
%
\mbox{[\#6]} \; P_{\rm rad} &\sim & 2.1 \times 10^{11} \,{\rm W}\,\left( \frac{\omega}{\omega _{\rm J}} \right)^{1/2} \left( \frac{M_p}{M_{\rm J}} \right)^{1/3} \left( \frac{a}{a_{\rm J}} \right)^{-4/3} \left( \frac{\dot M_\star}{\dot M_{\odot}} \right)^{2/3} \left( \frac{v}{v_{\odot}} \right)^{5/3} \, \\
%
\mbox{[\#7]} \; P_{\rm rad} &\sim & 2.1 \times 10^{11} \,{\rm W}\,\left( \frac{\omega}{\omega _{\rm J}} \right)^{2/3} \left( \frac{M_p}{M_{\rm J}} \right)^{1/3} \left( \frac{a}{a_{\rm J}} \right)^{-4/3} \left( \frac{\dot M_\star}{\dot M_{\odot}} \right)^{2/3} \left( \frac{v}{v_{\odot}} \right)^{5/3} \, \\
%
\mbox{[\#8]} \; P_{\rm rad} &\sim & ? \times 10^{11} \,{\rm W}\, \left( \frac{M_p}{M_{\rm J}} \right)^{0.69}  \left( \frac{t}{4.5 \rm~Gyr} \right)^{-0.288}  \left( \frac{a}{a_{\rm J}} \right)^{-4/3} \left( \frac{\dot M_\star}{\dot M_{\odot}} \right)^{2/3} \left( \frac{v}{v_{\odot}} \right)^{5/3} \, \\
\end{eqnarray}
%%%%%%%%%
For the last equation, we used the scaling of the luminosity $L$, equation (\ref{eq:burrowsLum}):
%%%
\begin{equation}
L \sim 1.84\times10^{-10} L_\odot \left( \frac{t}{4.5 \rm~Gyr} \right)^{-1.3} \left( \frac{M_p}{M_{\rm J}} \right)^{2.64} \tag{\ref{eq:burrowsLum}}
\end{equation} 
%%%

In summary, we write down the scaling of the radio emission as below, 
%%%
\begin{equation}
P_{\rm rad} \times 10^{11} \,{\rm W}\, \left( \frac{\omega}{\omega _{\rm J}} \right)^{\alpha } \left( \frac{M_p}{M_{\rm J}} \right)^{\beta }  \left( \frac{t}{4.5 \rm~Gyr} \right)^{-\gamma }  \left( \frac{a}{a_{\rm J}} \right)^{-\delta }  \left( \frac{\dot M_\star}{\dot M_{\odot}} \right)^{\epsilon } \left( \frac{v}{v_{\odot}} \right)^{\phi } \, \\
\end{equation} 
%%%
with $\alpha = 0.5-0.8$, $\beta = 0.33-1.33$, $\gamma = 0-0.288$, $\gamma = 1.17-1.60$, $\epsilon = 0.58-0.8$, $\phi = 1.46-2.0$. 


\end{document}

