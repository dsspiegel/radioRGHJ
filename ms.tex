\documentclass{emulateapj}

\usepackage{epsfig}
\usepackage{amsmath}
\usepackage{rotating}
\usepackage{natbib}
%\usepackage{lscape}
\usepackage{enumerate}
\usepackage{multirow}
\usepackage{array}
\usepackage{appendix}
\usepackage{comment}

\bibliographystyle{apj}

\def\plotonesc#1{\centering \leavevmode
\includegraphics[clip=, width=1.70\columnwidth]{#1}}
\def\plotoneh#1{\centering \leavevmode
\includegraphics[clip=, width=.95\columnwidth]{#1}}
\def\plotone#1{\centering \leavevmode
\includegraphics[clip=, width=.85\columnwidth]{#1}}
\def\plotoneShrinkSmall#1{\centering \leavevmode
\includegraphics[clip=, width=.49\columnwidth]{#1}}
\def\plotoneShrinkMed#1{\centering \leavevmode
\includegraphics[clip=, width=.55\columnwidth]{#1}}
\def\plotoneShrinkBig#1{\centering \leavevmode
\includegraphics[clip=, width=.65\columnwidth]{#1}}
\def\plottwo#1#2{\centering \leavevmode
\includegraphics[width=.45\columnwidth]{#1} \hfil
\includegraphics[width=.45\columnwidth]{#2}}
\def\plottwob#1#2{\centering \leavevmode
\includegraphics[width=.49\columnwidth]{#1} \hfil
\includegraphics[width=.49\columnwidth]{#2}}
\def\plottwor#1#2{\centering \leavevmode
\includegraphics[width=.55\columnwidth,angle=90]{#1} \hfil
\includegraphics[width=.55\columnwidth,angle=90]{#2}}
\def\plotthree#1#2#3{\centering \leavevmode
\includegraphics[width=.3\columnwidth]{#1} \hfil
\includegraphics[width=.3\columnwidth]{#2} \hfil
\includegraphics[width=.3\columnwidth]{#3}}

\newcommand{\cN}[1]{\mathcal{N}}
\newcommand{\pn}[1]{\mbox{$(#1)$}}
\newcommand{\spa}{\mbox{ }}
\def\gsim{\;\rlap{\lower 2.5pt
 \hbox{$\sim$}}\raise 1.5pt\hbox{$>$}\;}
\def\lsim{\;\rlap{\lower 2.5pt
   \hbox{$\sim$}}\raise 1.5pt\hbox{$<$}\;}

% set formatting properties
\setlength{\textwidth}{6.5in}
\setlength{\textheight}{8.8in}
\setlength{\hoffset}{0.0in}
\setlength{\voffset}{-0.4in}
%\setlength{\voffset}{0.3in}
\parindent 0.2in
\parskip 0.1in



%%%%%%%%%%%%%%%%%%%%%%%%%%%%%%%%%%%%%%%%%%%%%%%%%
% THE DOCUMENT BEGINS HERE                      %
%%%%%%%%%%%%%%%%%%%%%%%%%%%%%%%%%%%%%%%%%%%%%%%%%

%\slugcomment{Submitted to ApJ, 20 October 2011}

\begin{document}


%%% Begin front material
%\twocolumn[%%% Begin front material

\title{Red-Giant Hot Jupiters: Brilliant Radio Beacons}


\author{
%
David S. Spiegel\altaffilmark{1} \\
%
{\bf and some order:} \\
%
Nikku Madhusudhan\altaffilmark{?} [Add correct affiliation] \\
%
Mehrdad Mirbabayi\altaffilmark{1} \\
%
Aaron Parsons\altaffilmark{?} [Add correct affiliation] \\
%
Tony Mroczkowski\altaffilmark{?} [Add correct affiliation]
}

\affil{$^1$Astrophysics Department, Institute for Advanced Study,
  Princeton, NJ 08540}

\affil{$^?$Astronomy Department, University of Cambridge, UK}

\affil{$^?$Astronomy Department, UC Berkeley}

\affil{$^?$Naval Research Laboratory}


\vspace{0.5\baselineskip}

\email{
dave@ias.edu
}


\begin{abstract}
  Red-giant hot Jupiters are jovian planets orbiting red-giant-branch
  or asymptotic-giant-branch (AGB) stars.  Post-main-sequence stars
  lose mass at much higher rates than main-sequence stars.  A jovian
  planet passing through the dense winds of its AGB host can capture
  stellar wind in its magnetosphere.  The cyclotron frequency of
  electrons from the stellar wind accreting onto the planet scales as
  100~MHz~$(B/30 {\rm~Gauss})$.  Such a planet might generate a radio
  luminosity that would be visible from kiloparsec distances.
\end{abstract}


\keywords{planets and satellites: Jupiter --- Sun: evolution ---
  planetary systems --- radiative transfer --- stars: evolution ---
  stars: AGB and post-AGB}
%]%%% End front material


\section{Introduction}
\label{sec:intro}

% fix
When stars less than $\sim$8~$M_\sun$ evolve off the main sequence,
they go through stages on the red-giant branch (RGB) and the
asymptotic-giant branch (AGB) when their radii and luminosities
increase by orders of magnitude.  Jovian planets in orbit around such
stars can be transiently heated to hot-Jupiter temperatures
($\sim$1000~K or more) at distances out to tens of AU, depending on
the star's mass.  These planets' existence was predicted by
\citet{spiegel+madhusudhan2012} and we will refer to them as
``red-giant hot Jupiters'' (RGHJs).

Roughly 20\% of the more than 700 currently known
exoplanets\footnote{See online catalogs such as
  http://www.openexoplanetcatalogue.com/ \citep{rein2012},
  http://exoplanet.eu \citep{schneider_et_al2011}, or
  http://exoplanets.org \citep{wright_et_al2011} for up-to-date
  lists.} have masses greater than half of Jupiter's, orbital radii
greater than 1~AU, and will become hot Jupiters (i.e., for the present
purposes, this means they will receive at least as much irradiation as
the hot Neptune

The cyclotron frequency is
\begin{equation}
f_{\rm cyclotron} = \frac{eB}{2\pi m_e c} \approx 2.8 {\rm~MHz} \left( \frac{B}{1 \rm~G} \right) \, .
\label{eq:cyc}
\end{equation}
For magnetic field strengths in the vicinity of $\sim$25-50~Gauss (G),
the cyclotron frequency will be in the range $\sim$75-150~MHz.

\citep{spiegel2008}

\citep{lecavelier_et_al2013}

\citep{janhunen_et_al2003}

\citep{zarka1992, zarka1998}

\citep{farrell_et_al2004}

\citep{lazio+farrell2007}: likelihood function, tau Boo search

\citep{lecavelier_et_al2009}

\citep{spiegel2012}

\citep{nordhaus+spiegel2013}

\citep{spiegel+madhusudhan2012}

\citep{jiang+jin1996}: 12cm radio of Jupiter during Shoemake-Levy-9

\citep{morin2012, morin_et_al2013}

\citep{christensen_et_al2009, christensen2010}

\citep{saar2001}: Inverse rossby number scaling of magnetic field: $B
\sim 60 {\rm~G} \times Ro^{-1.2}$.  Here, he takes $Ro = \tau_c/P_{\rm
  rot}$, where $\tau_c$ is the convective turnover time = ?
\citep{gilliland1986}.  Well, $F \sim \rho v_c^3 \sim \rho
(H/\tau_c)^3$, so $\tau_c \sim H (\rho / F)^{1/3} = H / (\sigma T_{\rm
  eff}^4 / \rho)^{1/3}$.  Alternatively, $\tau_{\rm conv} \sim (M R^2
/ L)^{1/3}$.

On dimensional grounds, the convective turnover time goes as
$\tau_{\rm conv} \sim (M R^2 / L)^{1/3}$.  According to
\citet{burrows_et_al2001}, radius $R$ and luminosity scale with time
as $R \sim R_J$, where $R_J$ is the radius of Jupiter, and
\begin{equation}
\frac{L}{10^{-9} L_\odot} \sim \left( \frac{t}{1 \rm~Gyr} \right)^{-1.3} \left( \frac{M}{1~M_J} \right)^{2.64} \, .
\label{eq:burrowsLum}
\end{equation}
So,
%\begin{eqnarray}
%\nonumber \tau_{\rm conv} & \sim & 3 {\rm~days} \times \left( \frac{M}{M_J} \right)^{1/3} \left( \frac{R}{R_J} \right)^{2/3} \left( \frac{L}{L_\odot} \right)^{-1/3} \\
% & = & 
%\end{eqnarray}
\begin{eqnarray}
\nonumber \tau_c & \sim & 3 {\rm~hrs} \times \frac{\left( \frac{H}{100 \rm~km} \right) \left( \frac{\rho}{10^{-5} \rm~g~cm^{-3}} \right)^{1/3}}{\left( \frac{L}{10^{-9} L_\odot} \right)^{1/3}} \\
 & = & 3 {\rm~hrs} \times \frac{\left( \frac{H}{100 \rm~km} \right) \left( \frac{\rho}{10^{-5} \rm~g~cm^{-3}} \right)^{1/3}}{\left( \frac{M}{1 ~M_J} \right)^{0.88} \left( \frac{t}{1 \rm~Gyr} \right)^{0.43}} \\
\end{eqnarray}
So the Rossby number $Ro = \tau_c/P_{\rm rot}$ is
\begin{eqnarray}
Ro & = & \frac{\tau_c}{P_{\rm rot}} \\
 & = & \left( \frac{P_{\rm rot}}{3 \rm~hrs} \right)^{-1} \times \frac{\left( \frac{H}{100 \rm~km} \right) \left( \frac{\rho}{10^{-5} \rm~g~cm^{-3}} \right)^{1/3}}{\left( \frac{M}{1 ~M_J} \right)^{0.88} \left( \frac{t}{1 \rm~Gyr} \right)^{0.43}}
\end{eqnarray}

\citep{hallinan_et_al2013}

\citep{desch+kaiser1984} radiometric Bode's law

Noting that
\begin{equation}
\dot{M}_* = 4\pi r^2 \rho[r] v \, ,
\label{eq:mdot
}\end{equation}
so $\rho[r] = \dot{M}_*/(4\pi r^2 v)$,
\begin{eqnarray}
\frac{\rho v^2}{2} = \frac{B^2}{8\pi} \sim \frac{B_0^2}{8\pi} \left( \frac{d}{d_0} \right)^{-3} \, ,
\end{eqnarray}
where $B_0$ is the field strength at a distance $d_0$ from the planet.
\begin{eqnarray}
\frac{\rho v^2}{2} & \sim & \frac{B_0^2}{8\pi} \left( \frac{d}{d_0} \right)^{-6} \\
\frac{\dot{M}_* v}{8\pi a^2} & = & \frac{B_0^2}{8\pi} \left( \frac{d}{d_0} \right)^{-6} \\
\frac{\dot{M}_* v}{r^2} & = & B_0^2 \left( \frac{d}{d_0} \right)^{-6} \\
d^6 & = & \frac{d_0^6 B_0^2 r^2}{\dot{M}_* v} \\
d_A & = & d_0 \left( \frac{B_0^2 r^2}{\dot{M}_* v} \right)^{1/6} \\
  & \sim & 4 R_J \left( \frac{d_0}{R_J}\right) \left\{ \frac{\left( \frac{B}{10 \rm~G} \right)^2 \left( \frac{r}{5 \rm~AU} \right)^2}{\left( \frac{\dot{M}_*}{10^{-6} M_\odot/\rm yr} \right) \left( \frac{v}{20 \rm~km/s} \right)} \right\}^{1/6}
\label{eq:Chapman-Ferraro}
\end{eqnarray}
In the above, $d_A$ is the distance from the planet to the Alfven
point, where the magnetic energy density $u_B = B^2 / 8\pi$ equals the
kinetic energy density in the stellar wind $u_w = \rho v^2/2$, and
$B_0$ is the magnetic field strength at distance $d_0$.

The escape speed is
\begin{equation}
v_{\rm esc}^* = \sqrt{\frac{2 G M_*}{R_*}} \, ,
\end{equation}
so if the stellar wind speed is a factor $C$ times the escape speed, then ...

The power incident on the planet within $d_A$ is
\begin{eqnarray}
P_{\rm inc} & = & \frac{\rho v^3}{2} \times \pi d_A^2 \\
\frac{\rho v^2}{2} & \sim & \frac{B_0^2}{8\pi} \left( \frac{d}{d_0} \right)^{-6} \\
d_A^6 & = & d_0^6 \frac{B_0^2}{4\pi \rho v^2} \\
d_A^2 & = & d_0^2 \left( \frac{B_0^2}{4\pi \rho v^2}\right)^{1/3} \\
\pi d_A^2 \frac{\rho v^3}{2} & = & \frac{d_0^2}{2} \left( \frac{\pi^2 \rho^2 v^7 B_0^2}{4} \right)^{1/3} \\
P_{\rm inc} & = & d_0^2 \left( \frac{\pi^2 \rho^2 v^7 B_0^2}{32} \right)^{1/3} 
%\frac{\dot{M}_* v}{8\pi a^2} & = & \frac{B_0^2}{8\pi} \left( \frac{d}{d_0} \right)^{-6} \\
%\frac{\dot{M}_* v}{r^2} & = & B_0^2 \left( \frac{d}{d_0} \right)^{-6} \\
%d^6 & = & \frac{d_0^6 B_0^2 r^2}{\dot{M}_* v} \\
%d_A & = & d_0 \left( \frac{B_0^2 r^2}{\dot{M}_* v} \right)^{1/6} \\
%  & \sim & 4 d_0 \left\{ \frac{\left( \frac{B}{10 \rm~G} \right)^2 \left( \frac{r}{5 \rm~AU} \right)^2}{\left( \frac{\dot{M}_*}{10^{-6} M_\odot/\rm yr} \right) \left( \frac{v}{20 \rm~km/s} \right)} \right\}^{1/6}
\end{eqnarray}
Note that $\rho v = \dot{M}_*/4\pi r^2$.  Therefore,
\begin{eqnarray}
P_{\rm inc} & = & d_0^2 \left( \frac{\pi^2 \rho^2 v^7 B_0^2}{32} \right)^{1/3} \\
 & = & d_0^2 \left( \frac{\dot{M}_*^2 v^5 B_0^2}{512 r^4} \right)^{1/3} \\
 & = & \frac{d_0^2}{8} \left( \frac{\dot{M}_*^2 v^5 B_0^2}{r^4} \right)^{1/3} \\
\nonumber  & \sim & 2 \times 10^{18} {\rm~W} \left( \frac{d_0}{R_J} \right)^2 \left( \frac{\dot{M}_*}{10^{-5} M_\odot / {\rm yr}} \right)^2 \\
 & & \times \left( \frac{v}{20 \rm~km/s} \right)^5 \left( \frac{B_0}{10 \rm~G} \right)^2 \left( \frac{r}{5 \rm~AU} \right)^{-4}
\end{eqnarray}


The power incident on the planet is

%\citep{lunine_et_al1989} -- error?

% http://kiss.caltech.edu/workshops/magnetic2013/presentations/winterhalter.pdf

%ftp://ftp.iwf.oeaw.ac.at/pub/Scherf/PRE-CD%20V2/pre6/nigl.pdf


\section{Conclusions}
\label{sec:conc}

\vspace{0.5in}

%\acknowledgements

{\sc Acknowledgments}

We thank many people for useful discussions, in particular Tony
Mroczkowzki.  DSS gratefully acknowledges support from a fellowship
from the AMIAS group.  NM acknowledges support from [???].

\newpage

mehrdad
\newpage
\bibliography{biblio.bib}


\clearpage

\end{document}
