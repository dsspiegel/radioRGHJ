\documentclass{emulateapj}

\usepackage{epsfig}
\usepackage{amsmath}
\usepackage{rotating}
\usepackage{natbib}
%\usepackage{lscape}
\usepackage{enumerate}
\usepackage{multirow}
\usepackage{array}
\usepackage{appendix}
\usepackage{comment}
\usepackage{color}

\bibliographystyle{apj}

\def\memo#1{\color{red}$[${\bf #1}$]$ \color{black}}

\def\plotonesc#1{\centering \leavevmode
\includegraphics[clip=, width=1.70\columnwidth]{#1}}
\def\plotoneh#1{\centering \leavevmode
\includegraphics[clip=, width=.95\columnwidth]{#1}}
\def\plotone#1{\centering \leavevmode
\includegraphics[clip=, width=.85\columnwidth]{#1}}
\def\plotoneShrinkSmall#1{\centering \leavevmode
\includegraphics[clip=, width=.49\columnwidth]{#1}}
\def\plotoneShrinkMed#1{\centering \leavevmode
\includegraphics[clip=, width=.55\columnwidth]{#1}}
\def\plotoneShrinkBig#1{\centering \leavevmode
\includegraphics[clip=, width=.65\columnwidth]{#1}}
\def\plottwo#1#2{\centering \leavevmode
\includegraphics[width=.45\columnwidth]{#1} \hfil
\includegraphics[width=.45\columnwidth]{#2}}
\def\plottwob#1#2{\centering \leavevmode
\includegraphics[width=.49\columnwidth]{#1} \hfil
\includegraphics[width=.49\columnwidth]{#2}}
\def\plottwor#1#2{\centering \leavevmode
\includegraphics[width=.55\columnwidth,angle=90]{#1} \hfil
\includegraphics[width=.55\columnwidth,angle=90]{#2}}
\def\plotthree#1#2#3{\centering \leavevmode
\includegraphics[width=.3\columnwidth]{#1} \hfil
\includegraphics[width=.3\columnwidth]{#2} \hfil
\includegraphics[width=.3\columnwidth]{#3}}

\newcommand{\cN}[1]{\mathcal{N}}
\newcommand{\pn}[1]{\mbox{$(#1)$}}
\newcommand{\spa}{\mbox{ }}
\def\gsim{\;\rlap{\lower 2.5pt
 \hbox{$\sim$}}\raise 1.5pt\hbox{$>$}\;}
\def\lsim{\;\rlap{\lower 2.5pt
   \hbox{$\sim$}}\raise 1.5pt\hbox{$<$}\;}

% set formatting properties
\setlength{\textwidth}{6.5in}
\setlength{\textheight}{8.8in}
\setlength{\hoffset}{0.0in}
\setlength{\voffset}{-0.4in}
%\setlength{\voffset}{0.3in}
\parindent 0.2in
\parskip 0.1in



%%%%%%%%%%%%%%%%%%%%%%%%%%%%%%%%%%%%%%%%%%%%%%%%%
% THE DOCUMENT BEGINS HERE                      %
%%%%%%%%%%%%%%%%%%%%%%%%%%%%%%%%%%%%%%%%%%%%%%%%%

%\slugcomment{Submitted to ApJ, 20 October 2011}

\begin{document}


%%% Begin front material
%\twocolumn[%%% Begin front material

\title{Red-Giant Hot Jupiters: Brilliant Radio Emitter?}


\author{
%
David S. Spiegel\altaffilmark{1, 2} \\
%
{\bf and some order:} \\
%
Jason Nordhaus\altaffilmark{3} \\
%
Nikku Madhusudhan\altaffilmark{4} \\
%
Mehrdad Mirbabayi\altaffilmark{1} \\
%
Aaron Parsons\altaffilmark{5} \\
%
Tony Mroczkowski\altaffilmark{6} \\
%
Neil Zimmerman\altaffilmark{7}
}

\affil{$^1$Astrophysics Department, Institute for Advanced Study,
  Princeton, NJ 08540}

\affil{$^2$Research \& Development, Project Florida Labs,
  New York, NY  10001}

\affil{$^2$Department of Mathematics, Rochester Institute of Technology}

\affil{$^3$Astronomy Department, University of Cambridge, UK}

\affil{$^4$Astronomy Department, UC Berkeley}

\affil{$^5$Naval Research Laboratory}

\affil{$^6$Department of Mechanical and Aerospace Engineering, Princeton University, Princeton, NJ 08544}


\vspace{0.5\baselineskip}

\email{
dave@ias.edu
}


\begin{abstract}
  Red-giant hot Jupiters are jovian planets orbiting red-giant-branch
  or asymptotic-giant-branch (AGB) stars.  Post-main-sequence stars
  lose mass at much higher rates than main-sequence stars.  A jovian
  planet passing through the dense winds of its AGB host can capture
  stellar wind in its magnetosphere.  The cyclotron frequency of
  electrons from the stellar wind accreting onto the planet scales as
  100~MHz~$(B/30 {\rm~Gauss})$.  Such a planet might generate a radio
  luminosity that would be visible from kiloparsec distances.
\end{abstract}


\keywords{planets and satellites: Jupiter --- Sun: evolution ---
  planetary systems --- radiative transfer --- stars: evolution ---
  stars: AGB and post-AGB}
%]%%% End front material


\section{Introduction}
\label{sec:intro}

Planets with strong magnetic fields may generate radio or X-ray emission when interacting with energetic charged particles. 
It has been known that Jupiter emits radio wave around 20 MHz due to the interaction with plasmas from the satellite Io \memo{?}, and 100-3000 MHz, which is cyclotron radiation from electrons that are accelerated in Jupiter's magnetic field \memo{?}. 
Similarly, large exoplanets can emit radio waves, depending on their intrinsic magnetic fields and the density of surrounding plasmas, e.g. stellar wind particles and particles from Io-like moons. 
Radio emission from exoplanets have been estimated taking account of several possible processes \citep{griebmeier2007}, and the search for these radio signatures are being conducted. 
%In general, there are four proposed models for a planet to emit radio wave \citep{griebmeier2007}: 1) the magnetic energy model, 2) kinetic energy model, 3) CME model, and 4) the unipolar interaction model. 
%The search for these radio emissions from extrasolar Jovian planets has been performed. 
While some indications were obtained \memo{?} \citep{lecavelier_et_al2013},  there is no clear detection claimed so far. 


%It has been predicted that
% fix
When stars less than $\sim$8~$M_\sun$ evolve off the main sequence,
they go through stages on the red-giant branch (RGB) and the
asymptotic-giant branch (AGB) when their radii and luminosities
increase by orders of magnitude. 



Not only luminous, red giants lose their masses at high rate, typically $\sim  10^{-8} M_{\odot }$/yr \citep{judge1991} for RGs \memo{How these values are observationally obtained?}. \memo{Dave mentioned $\sim  10^{-5} M_{\odot }$/yr, where does this come from?} through massive (but slow) stellar wind \citep{suzuki2008} \memo{cite more papers?}. 
Jovian planets in orbit around such
stars can be transiently heated to hot-Jupiter temperatures
($\sim$1000~K or more) at distances out to tens of AU, depending on
the star's mass. These planets' existence was predicted by
\citet{spiegel+madhusudhan2012} and we will refer to them as
``red-giant hot Jupiters'' (RGHJs). 
When massive stellar wind reach the planetary magnetosphere standoff points  \memo{??}, they presumably deposit part of their energy generating bright radio emission. 

In this paper, we examine the radio brightness of RGHJs, taking account of the increased stellar wind. 




%Red giants lose their masses at a high rate, typically $\sim  10^{-5} M_{\odot }/yr $ through massive stellar wind. These stellar wind particle can 


%Roughly 20\% of the more than 700 \memo{check!} currently known exoplanets around main-sequence stars \footnote{See online catalogs such as http://www.openexoplanetcatalogue.com/ \citep{rein2012}, http://exoplanet.eu \citep{schneider_et_al2011}, or http://exoplanets.org \citep{wright_et_al2011} for up-to-date lists.} have masses greater than half of Jupiter's, orbital radii greater than 1~AU, and will become hot Jupiters (i.e., for the present purposes, this means they will receive at least as much irradiation as the currently known hot Jupiters/Neptunes \citep{spiegel+madhusudhan2012}. 



\section{Assumpsions}

\subsection{Mechanisms}


\subsection{Properties for Stellar Wind of RGs}

The mass loss rate of the stellar wind can be estimated from observation \memo{How?}. For the known red giants, the values are typically $\dot M \sim 10^{-8} M_{\odot}$/yr, and can be as much as $\sim  10^{-5} M_{\odot }$/yr for AGB stars, significantly greater than the solar mass-loss rate: $\sim 10^{-14} M_{\odot}$/yr. 
\begin{equation}
\frac{\dot M}{\dot M_{\odot}} \sim 10^6
\end{equation}

The stellar wind velocity becomes smaller because of the small escape velocity (due to expanded stellar radius)\footnote{Escape velocity is $\sqrt{2GM/r}$.}. Assuming that the stellar wind scales as the escape velocity, RGs with radius $R=100R_{\odot}$ have 10 times as slow stellar wind as that at the main sequence.
\begin{equation}
\frac{v}{v_{\odot}} \sim 10^{-1}
\end{equation}

The temperature of stellar wind of RGs is expected to be 2 orders of magnitude lower than their main sequence counterparts, mainly because the sound speed of hot corona exceeds the escape speed, i.e., hot corona cannot be confined in the stellar atmosphere \citep{suzuki2008}. 
\begin{equation}
\frac{T}{T_{\odot}} \sim 10^{-2}
\end{equation}


\subsection{Properties for Planetary Magnetic Field}

Theoretically, the magnetic moment of gaseous planets are expressed with the following scaling relationship \citep{griebmeier2004}:
%%%%%%%%%% 
\begin{equation}
\mathcal{M} \propto  \omega ^{\alpha } \rho_c ^{\beta } r_c^{\gamma } \sigma ^{\delta }
\end{equation}
%%%%%%%%%%
where $\omega $ is the spin angular velocity, $\rho _c$, $r_c$ and $\sigma $ are the density, the radius, and the conductivity of the ``dynamo region'' where the density is high enough for hydrogen to be metallic, respectively. 
The scaling indexes are estimated to be $\alpha \sim 1/2-1$, $\beta \sim 1/2$, $\gamma \sim 3-4$, and $\sigma \sim -1/2-0$. In this paper, we assume $\alpha =1$, $\beta =1/2$, and $r_c = 7/2$ \citep{sano1993}. \memo{validity?}

Unlike usual hot jupiters, RGHJs are not subject to tidal lock, because the gravitational effects of their host star does not change even if the star evolves into red giants. Without no physical insights of the typical spin period, we simply assume that of Jupiter: $\omega = 9.925$ [hr]. 

In order to evaluate $\rho _c $ and $r_c$, we need a model of internal structure of gaseous planets. We simply use the same model as \citet{griebmeier2004}, where the density as a function of radius from the center is described as:
%%%%%%%%%% 
\begin{equation}
\rho (r) = \left( \frac{\pi M_p}{4 R_p^3} \right) \frac{\sin \left( \pi \frac{r}{R_p} \right)}{\left( \pi \frac{r}{R_p} \right)} \label{eq:rho_r}
\end{equation}
%%%%%%%%%%
and assume that the hydrogen becomes metallic when $\rho (r)$ exceeds the critical density $\rho_c=700\,\mbox{kg/m}^3$. The density of the metallic core is obtained by averaging the density in the core. 

We assume that the conductivity $\sigma $ is the same as Jupiter \memo{?}. 

\section{Estimates of Radio Emission from RGHJ}

\subsection{Estimates of frequency}

The cyclotron frequency is
\begin{eqnarray}
f_{\rm cyclotron} &=& \frac{eB}{2\pi m_e c} \approx 28 {\rm~MHz} \left( \frac{B}{10 \rm~G} \right) \, ,
\label{eq:cyc} \\
B &=& \frac{\mu_0}{2\pi}\frac{\mathcal{M}}{R_p^3} \\
&\sim & 9.1 \mbox{[G]} \left( \frac{\mathcal{M}}{\mathcal{M}_J} \right) \left( \frac{R_p}{R_{p, J}} \right)^{-3}. 
\end{eqnarray}
%where $\mathcal{M}_J = 1.56 \times 10^{27} \mbox{A m}^2$. 
For magnetic field strengths in the vicinity of $\sim$25-50~Gauss (G),
the cyclotron frequency will be in the range $\sim$75-150~MHz.

Since the 
Figure \ref{fig:f_cyclotron} shows 

The emission is observable only when the maximum frequency is larger than the plasma frequency of the surrounding stellar wind. 
The plasma frequency is 
\begin{eqnarray}
f_{\rm plasma} &=& \frac{1}{2\pi} \sqrt{\frac{ne^2}{\epsilon _0 m_e}} \\
&\approx & 4\mbox{[MHz]} \left( \frac{\dot M}{10^6 \dot M_{\odot}}\right)^{1/2} \left(\frac{v}{10^{-1} v_{\odot}}  \right)^{-1/2}
\end{eqnarray}

...OK!\memo{double-check}

\subsection{Estimates of intensity}

According to \citet{griebmeier2007} \memo{The following equations are quoted from the paper. Needs to elaborate. }
\begin{enumerate}
\item kinetic energy model
\begin{equation}
P_{\rm inp} \propto n v^3 r_s ^2 \label{eq:Pinp_kin}
\end{equation}
\item magnetic energy model:
\begin{equation}
P_{\rm inp} \propto v B_{\bot }^2 r_s ^2 \label{eq:Pinp_mag}
\end{equation}
\item CME model\\
Not sure about CME for red giants. We don't consider?
\item the unipolar interaction model\\
??
\end{enumerate}
where $n$ and $v$ are density and velocity of stellar wind, respectively,  $ B_{\bot }$ is the interstellar magnetic field perpendicular to the stellar wind flow, and $r_s$ is the radius of magnetic standoff point. 

$r_s$ is determined by the balance between the magnetic pressure of planetary magnetosphere and kinetic pressure of stellar wind\footnote{Precisely, the magnetic standoff point is described by equation (9) of \citet{griebmeier2007}, which is. 
\begin{equation}
r_s = \left[ \frac{\mu_0 f_0^2 \mathcal{M}}{8\pi^2 (m_p n(d) v(d)^2 + 2 n(d) k_B T)} \right]^{1/6}
\end{equation}
where $\mathcal{M}$ is the planetary magnetic moment and $m_p$ is the proton mass. 
In the following arguments based on Dave's sketch, the second term in the denominator is ignored. This is valid for Solar wind where the first term is larger than the second term by 2 orders of magnitude. \memo{check!}}. Thus, 
\begin{equation}
\frac{m_p n v ^2}{2} = \frac{B^2}{2\pi}\left( \frac{r_s}{R_p} \right)^{-6} 
\end{equation}
Therefore,
\begin{eqnarray}
r_s &=& R_p \left( \frac{B^2}{\pi \dot M v (1/4\pi d^2)} \right)^{1/6} = R_p \left( \frac{4 B^2 d^2}{\dot M v } \right)^{1/6} \\
&\approx & 5 R_J \left(\frac{B}{10 \mbox{[G]}}\right)^{1/3} \left(\frac{\dot M}{10^6 \dot M_{\odot}}\right)^{-1/6}\\
&& \;\;\;\;\; \;\;\;\;\; \times \left(\frac{v}{10^{-1} v_{\odot}}\right)^{-1/6}  \left(\frac{d}{5 \mbox{[AU]}}\right)^{1/3} 
\end{eqnarray}

Therefore, the radio emission expected from kinetic energy model (equation (\ref{eq:Pinp_kin})) is 
\begin{eqnarray}
P_{\rm radio} &=& P_{\rm radio, J} \left(\frac{\dot M }{\dot M_{\odot }} \right) \left(\frac{v}{v_{\odot}} \right) ^2 \left(\frac{d}{5 \rm{AU}} \right) ^{-2}  \left(\frac{r_s}{r_{s, {\rm J}}} \right)^2 \\
&=& P_{\rm radio, J} \left(\frac{\dot M }{\dot M_{\odot }} \right)^{\frac{2}{3}} \left(\frac{v}{v_{\odot}} \right) ^{\frac{5}{3}} \left(\frac{d}{5 \rm{AU}} \right) ^{-2} \left(\frac{B}{B_J} \right)^{\frac{2}{3}} \\
&\approx& 200 P_{\rm radio, J} \left(\frac{\dot M }{10^6 \dot M_{\odot }} \right)^{\frac{2}{3}} \left(\frac{v}{10^{-1} v_{\odot}} \right) ^{\frac{5}{3}} \\
&& \;\;\;\;\; \;\;\;\;\; \times \left(\frac{d}{5 \rm{[AU]}} \right) ^{-4/3} \left(\frac{B}{10 \mbox{[G]}} \right)^{\frac{2}{3}} \
\end{eqnarray}


\memo{Following arguments are based on \citep{griebmeier2007}. }
Assuming $P_{\rm inp, J} = 2.1\times 10^{11}$ W \citep{griebmeier2007}, observed RGHJ radio emission at distance $l$ is
\begin{eqnarray}
P_{\rm obs} &=& \frac{P_{\rm radio}}{\Omega l^2 f_{\rm cyclotron}} \\
&\approx & 5\times 10^{-6} \mbox{Jy} \left( \frac{P_{\rm radio}}{P_{\rm radio, J}} \right) \left( \frac{l}{10\mbox{[pc]}} \right)^{-2} \left( \frac{B}{10\mbox{[G]}} \right) ^{-1} \\
&\approx & 10^{-3} \mbox{Jy} \left( \frac{P_{\rm radio}}{200 P_{\rm radio, J}} \right) \left( \frac{l}{10\mbox{[pc]}} \right)^{-2} \left( \frac{B}{10\mbox{[G]}} \right) ^{-1}
\end{eqnarray}
where $\Omega $ is the solid angle of emission, which is assumed $\Omega = 1.6$. 




\subsection{Will the electron spiraling and offset the magnetic field? (Mehrdad Mirababayi)}


%%%%%%%%%%%%%%%%%%%%%%%%%%%%%%%%%%%%%%%%%%%%%%%%%%%%%%%%%%%%%%%%%%%%%%%%
\newpage
\section{Discussion on Observability}

\subsection{Radio emission from red giants stars}

(Jason)


%%%%%%%%%%%%%%%%%%%%%%%%%%%%%%%%%%%%%%%%%%%%%%%%%%%%%%%%%%%%%%%%%%%%%%%%

\citet{gorman2013}


\subsection{Expected population of RGHJ in the observable volume}




\section{Dave's sketch}

\citep{spiegel2008}

\citep{lecavelier_et_al2013}

\citep{janhunen_et_al2003}

\citep{zarka1992, zarka1998}

\citep{farrell_et_al2004}

\citep{lazio+farrell2007}: likelihood function, tau Boo search

\citep{lecavelier_et_al2009}

\citep{spiegel2012}

\citep{nordhaus+spiegel2013}



\citep{jiang+jin1996}: 12cm radio of Jupiter during Shoemake-Levy-9

\citep{morin2012, morin_et_al2013}

\citep{christensen_et_al2009, christensen2010}

\citep{saar2001}: Inverse rossby number scaling of magnetic field: $B
\sim 60 {\rm~G} \times Ro^{-1.2}$.  Here, he takes $Ro = \tau_c/P_{\rm
  rot}$, where $\tau_c$ is the convective turnover time = ?
\citep{gilliland1986}.  Well, $F \sim \rho v_c^3 \sim \rho
(H/\tau_c)^3$, so $\tau_c \sim H (\rho / F)^{1/3} = H / (\sigma T_{\rm
  eff}^4 / \rho)^{1/3}$.  Alternatively, $\tau_{\rm conv} \sim (M R^2
/ L)^{1/3}$.

On dimensional grounds, the convective turnover time goes as
$\tau_{\rm conv} \sim (M R^2 / L)^{1/3}$.  According to
\citet{burrows_et_al2001}, radius $R$ and luminosity scale with time
as $R \sim R_J$, where $R_J$ is the radius of Jupiter, and
\begin{equation}
\frac{L}{10^{-9} L_\odot} \sim \left( \frac{t}{1 \rm~Gyr} \right)^{-1.3} \left( \frac{M}{1~M_J} \right)^{2.64} \, .
\label{eq:burrowsLum}
\end{equation}
So,
%\begin{eqnarray}
%\nonumber \tau_{\rm conv} & \sim & 3 {\rm~days} \times \left( \frac{M}{M_J} \right)^{1/3} \left( \frac{R}{R_J} \right)^{2/3} \left( \frac{L}{L_\odot} \right)^{-1/3} \\
% & = & 
%\end{eqnarray}
\begin{eqnarray}
\nonumber \tau_c & \sim & 3 {\rm~hrs} \times \frac{\left( \frac{H}{100 \rm~km} \right) \left( \frac{\rho}{10^{-5} \rm~g~cm^{-3}} \right)^{1/3}}{\left( \frac{L}{10^{-9} L_\odot} \right)^{1/3}} \\
 & = & 3 {\rm~hrs} \times \frac{\left( \frac{H}{100 \rm~km} \right) \left( \frac{\rho}{10^{-5} \rm~g~cm^{-3}} \right)^{1/3}}{\left( \frac{M}{1 ~M_J} \right)^{0.88} \left( \frac{t}{1 \rm~Gyr} \right)^{0.43}} \\
\end{eqnarray}
So the Rossby number $Ro = \tau_c/P_{\rm rot}$ is
\begin{eqnarray}
Ro & = & \frac{\tau_c}{P_{\rm rot}} \\
 & = & \left( \frac{P_{\rm rot}}{3 \rm~hrs} \right)^{-1} \times \frac{\left( \frac{H}{100 \rm~km} \right) \left( \frac{\rho}{10^{-5} \rm~g~cm^{-3}} \right)^{1/3}}{\left( \frac{M}{1 ~M_J} \right)^{0.88} \left( \frac{t}{1 \rm~Gyr} \right)^{0.43}}
\end{eqnarray}

\citep{hallinan_et_al2013}

\citep{desch+kaiser1984} radiometric Bode's law

Noting that
\begin{equation}
\dot{M}_* = 4\pi r^2 \rho[r] v \, ,
\label{eq:mdot
}\end{equation}
so $\rho[r] = \dot{M}_*/(4\pi r^2 v)$,
\begin{eqnarray}
\frac{\rho v^2}{2} = \frac{B^2}{8\pi} \sim \frac{B_0^2}{8\pi} \left( \frac{d}{d_0} \right)^{-3} \, ,
\end{eqnarray}
where $B_0$ is the field strength at a distance $d_0$ from the planet.
\begin{eqnarray}
\frac{\rho v^2}{2} & \sim & \frac{B_0^2}{8\pi} \left( \frac{d}{d_0} \right)^{-6} \\
\frac{\dot{M}_* v}{8\pi a^2} & = & \frac{B_0^2}{8\pi} \left( \frac{d}{d_0} \right)^{-6} \\
\frac{\dot{M}_* v}{r^2} & = & B_0^2 \left( \frac{d}{d_0} \right)^{-6} \\
d^6 & = & \frac{d_0^6 B_0^2 r^2}{\dot{M}_* v} \\
d_A & = & d_0 \left( \frac{B_0^2 r^2}{\dot{M}_* v} \right)^{1/6} \\
  & \sim & 4 R_J \left( \frac{d_0}{R_J}\right) \left\{ \frac{\left( \frac{B}{10 \rm~G} \right)^2 \left( \frac{r}{5 \rm~AU} \right)^2}{\left( \frac{\dot{M}_*}{10^{-6} M_\odot/\rm yr} \right) \left( \frac{v}{20 \rm~km/s} \right)} \right\}^{1/6}
\label{eq:Chapman-Ferraro}
\end{eqnarray}
In the above, $d_A$ is the distance from the planet to the Alfven
point, where the magnetic energy density $u_B = B^2 / 8\pi$ equals the
kinetic energy density in the stellar wind $u_w = \rho v^2/2$, and
$B_0$ is the magnetic field strength at distance $d_0$.

The escape speed is
\begin{equation}
v_{\rm esc}^* = \sqrt{\frac{2 G M_*}{R_*}} \, ,
\end{equation}
so if the stellar wind speed is a factor $C$ times the escape speed, then ...

The power incident on the planet within $d_A$ is
\begin{eqnarray}
P_{\rm inc} & = & \frac{\rho v^3}{2} \times \pi d_A^2 \\
\frac{\rho v^2}{2} & \sim & \frac{B_0^2}{8\pi} \left( \frac{d}{d_0} \right)^{-6} \\
d_A^6 & = & d_0^6 \frac{B_0^2}{4\pi \rho v^2} \\
d_A^2 & = & d_0^2 \left( \frac{B_0^2}{4\pi \rho v^2}\right)^{1/3} \\
\pi d_A^2 \frac{\rho v^3}{2} & = & \frac{d_0^2}{2} \left( \frac{\pi^2 \rho^2 v^7 B_0^2}{4} \right)^{1/3} \\
P_{\rm inc} & = & d_0^2 \left( \frac{\pi^2 \rho^2 v^7 B_0^2}{32} \right)^{1/3} 
%\frac{\dot{M}_* v}{8\pi a^2} & = & \frac{B_0^2}{8\pi} \left( \frac{d}{d_0} \right)^{-6} \\
%\frac{\dot{M}_* v}{r^2} & = & B_0^2 \left( \frac{d}{d_0} \right)^{-6} \\
%d^6 & = & \frac{d_0^6 B_0^2 r^2}{\dot{M}_* v} \\
%d_A & = & d_0 \left( \frac{B_0^2 r^2}{\dot{M}_* v} \right)^{1/6} \\
%  & \sim & 4 d_0 \left\{ \frac{\left( \frac{B}{10 \rm~G} \right)^2 \left( \frac{r}{5 \rm~AU} \right)^2}{\left( \frac{\dot{M}_*}{10^{-6} M_\odot/\rm yr} \right) \left( \frac{v}{20 \rm~km/s} \right)} \right\}^{1/6}
\end{eqnarray}
Note that $\rho v = \dot{M}_*/4\pi r^2$.  Therefore,
\begin{eqnarray}
P_{\rm inc} & = & d_0^2 \left( \frac{\pi^2 \rho^2 v^7 B_0^2}{32} \right)^{1/3} \\
 & = & d_0^2 \left( \frac{\dot{M}_*^2 v^5 B_0^2}{512 r^4} \right)^{1/3} \\
 & = & \frac{d_0^2}{8} \left( \frac{\dot{M}_*^2 v^5 B_0^2}{r^4} \right)^{1/3} \\
\nonumber  & \sim & 2 \times 10^{18} {\rm~W} \left( \frac{d_0}{R_J} \right)^2 \left( \frac{\dot{M}_*}{10^{-5} M_\odot / {\rm yr}} \right)^2 \\
 & & \times \left( \frac{v}{20 \rm~km/s} \right)^5 \left( \frac{B_0}{10 \rm~G} \right)^2 \left( \frac{r}{5 \rm~AU} \right)^{-4}
\end{eqnarray}


The power incident on the planet is

%\citep{lunine_et_al1989} -- error?

% http://kiss.caltech.edu/workshops/magnetic2013/presentations/winterhalter.pdf

%ftp://ftp.iwf.oeaw.ac.at/pub/Scherf/PRE-CD%20V2/pre6/nigl.pdf


%%%%%%%%%%%%%%%%%%%%%%%%%%%%%%%%%%%%%
\subsection{}

%%%%%%%%%%%%%%%%%%%%%%%%%%%%%%%%%%%%%%%%%%%%%%%%%%%%%%%%%%%%%%%%%%%%%%%%
\section{Conclusions}
\label{sec:conc}

\vspace{0.5in}

%\acknowledgements

{\sc Acknowledgments}

We thank many people for useful discussions, in particular Tony
Mroczkowzki.  DSS gratefully acknowledges support from a fellowship
from the AMIAS group.  NM acknowledges support from [???].

\newpage

mehrdad

\newpage

\bibliography{biblio.bib}


\clearpage

\end{document}
